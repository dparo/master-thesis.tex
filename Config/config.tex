%%
%% Main configuration flags
%%
\setbool{DebugBuild}{false}
\setbool{GenerateTableOfContents}{true}
\setbool{GenerateListOfFigures}{true}
\setbool{GenerateListOfTables}{true}


%%
%% Customize your work
%%
\newcommand{\ThemeName}{draft}
\newcommand{\UnivName}{University of Padova}
\newcommand{\UnivPlace}{Padova}
\newcommand{\DeptName}{Information Engineering}
\definecolor{M2}{HTML}{9B0014}
\definecolor{SchoolColor}{rgb}{0.71, 0, 0.106} %Unipd color


\newcommand{\AcademicYear}{2020-2021}
\newcommand{\Date}{Jun, 2022}

\newcommand{\AuthorName}{Davide}
\newcommand{\AuthorSurname}{Paro}

\newcommand{\DegreeName}{Master Degree in Computer Engineering}
\newcommand{\Title}{A MIP based approach for the Capacitated Profitable Tour Problem (CPTP)}

\newcommand{\SupName}{Domenico}
\newcommand{\SupSurname}{Salvagnin}

%comment these lines if there are no supervisors
\newcommand{\CosupName}{Roberto}
\newcommand{\CosupSurname}{Roberti}





%%
%% Customize part of the look-and-feel
%%

\chapterstyle{hansen}
\linespread{1.25}             % About 1.5 spacing in Word
\setlength{\parindent}{0pt}   % No paragraph indents
\setlength{\parskip}{1em}     % Paragraphs separated by one line


% How much section depth is actually gathered and shown in the table of contents
\setcounter{tocdepth}{2}       % 1 = show only section, 2 = show subsection, 3 = show subsubsection

% To which maximum section depth, latex produces a section numbering
\setcounter{secnumdepth}{5}       % For the Book document class
\setsecnumdepth{subsubsection}    % For the Memoir document class

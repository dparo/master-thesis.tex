%
%% "Core" packages
%%
\usepackage{afterpackage}        % To run commands conditionally after a package is loaded (if it is loaded)
\usepackage[htt]{hyphenat}       % To enable hyphenation and provides hyphenatable monospaced fonts, i.e. \texttt{} reflows on new lines correctly
\usepackage[pdfa,pdfusetitle]{hyperref}      % To allow for interactive links (click with the mouse). PDFA option to annotate links in conformance with PDF-A spec
\usepackage{graphicx}            % Allows you to insert figures
\usepackage{amsmath}             % Allows you to do equations
\usepackage{amssymb}
\usepackage{mathtools}           % mathtools builds on top of asmmath
\usepackage{accents}
\usepackage{csquotes}
\usepackage{titling}
\usepackage{url}       % For typesetting urls
\usepackage{subcaption}    % For handling subfigures, i.e. nesting figures, list of figures, array of figures, etc
\usepackage{booktabs}     % The package enhances the quality of tables in LaTeX, providing extra commands as well as behind-the-scenes optimisation.
\usepackage{multirow}

\usepackage{cleveref}            % To allow multiple cross-references to equations inside the same cref{} command
\usepackage{soul}

\usepackage{pdflscape}

%% Generate PDF/A-1 conformant document
\usepackage[\PdfXConformanceStandard,mathxmp]{pdfx}
\usepackage{colorprofiles}

% biblatex is better than default bibtex
\usepackage[
	backend=biber,
	natbib,       % natbib compatibility: Create natbib aliases commands that map to biblatex core commands
	style=authoryear,
	bibstyle=numeric,
	mincitenames=1,
	maxcitenames=1,
	maxbibnames=99,
	hyperref=true,
	backref=true,
	sorting=none,              % none = sort based on citation appearance order in the document.
	sortcites=false,           % Whether or not to sort citations if multiple entry keys are passed to a citation command.
	% Fix citation uniqueness. Biblatex is (by default) allowed to override maxcitenames if it deems this to provide a more unique authorlist.
	uniquelist=false,
	uniquename=false,
]{biblatex}

%%
%% ""Debug"" packages
%%
\usepackage{todonotes}        % For writing todos and notes in yor pdf
\usepackage{lipsum}           % For Lorem Ipsum
\usepackage{pgfornament}

\ifbool{DebugBuild}{
	\usepackage{showframe}        % To debug the frame border, padding, etc. It draws a box around each major region of the page
	\usepackage{layout}           % To generate automatically a layout page
}{
	% Else clause
}

%%
%% User level packages, can be customized to fit your need
%%
\usepackage{float}            % use [H] to force figure position exactly as placed in the original latex source
\usepackage{fancyvrb}         % Fancy handling of verbatim blocks. It allows many features. One of such features it allows to center the verbatim blocks
\usepackage{listings}         % For verbatim source code highlighting
\usepackage[linesnumbered, ruled]{algorithm2e}     % To write pseudocode algorithms in a consistent Latex-alike style

\usepackage{tikz}            % A drawing package to draw arbitrary shapes, useful even for drawing networks/graphs

% To access various style for arrow heads
\usetikzlibrary{snakes,arrows,shapes}

\newcommand{\crefrangeconjunction}{--}

% Instruct cref on how to refer to algorithms references
\crefname{algocf}{algorithm}{algorithms}
\Crefname{algocf}{Algorithm}{Algorithms}

\definecolor{webgreen}{rgb}{0,.5,0}
\definecolor{webbrown}{rgb}{.6,0,0}
\definecolor{webblue}{rgb}{0,0,0.8}

\ifbool{PressPrint}{
	\colorlet{defaultcolor}{.}

	% Optimize for pressprints by using black color
	% for the clickable hyperrefs as to reduce the number
	% of pages containing color.
	\colorlet{urlcolor}{defaultcolor}
	\colorlet{linkcolor}{defaultcolor}
	\colorlet{citecolor}{defaultcolor}
}{
	\colorlet{urlcolor}{webbrown}
	\colorlet{linkcolor}{webbrown}
	\colorlet{citecolor}{webblue}
}

\ifbool{GenerateGlossary}{
	\usepackage[acronym]{glossaries} % For list of acronyms
}{}

\captionsetup{
	tableposition=top,
	figureposition=bottom,
	font=small,
	format=hang,
	labelfont=bf
}

\hypersetup{
	%hyperfootnotes=false,
	%pdfpagelabels,
	%draft,	% = elimina tutti i link (utile per stampe in bianco e nero)
	colorlinks=true,
	linktocpage=true,
	pdfstartpage=1,
	pdfstartview=FitV,
	% decommenta la riga seguente per avere link in nero (per esempio per la
	%stampa in bianco e nero)
	%colorlinks=false, linktocpage=false, pdfborder={0 0 0}, pdfstartpage=1,
	%pdfstartview=FitV,
	breaklinks=true,
	pdfpagemode=UseNone,
	pageanchor=true,
	pdfpagemode=UseOutlines,
	plainpages=false,
	bookmarksnumbered,
	bookmarksopen=true,
	bookmarksopenlevel=1,
	hypertexnames=true,
	pdfhighlight=/O,
	%nesting=true,
	%frenchlinks,
	urlcolor=urlcolor,
	linkcolor=linkcolor,
	citecolor=citecolor,
	%pagecolor=RoyalBlue,
	%urlcolor=Black, linkcolor=Black, citecolor=Black, %pagecolor=Black,
	pdfencoding=unicode,
	pdftitle={\Title},
	pdfauthor={\AuthorName\ \AuthorSurname},
	pdfsubject={},
	pdfkeywords={\Keywords},
	pdfcreator={pdfLaTeX},
	pdfproducer={LaTeX}
}

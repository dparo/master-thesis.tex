%
%% "Core" packages
%%
\usepackage{ifthen}              % To write sane if-then statemnts
\usepackage{afterpackage}        % To run commands conditionally after a package is loaded (if it is loaded)
\usepackage[htt]{hyphenat}       % To enable hyphenation and provides hyphenatable monospaced fonts, i.e. \texttt{} reflows on new lines correctly
\usepackage{hyperref}            % To allow for interactive links (click with the mouse)
\usepackage{graphicx}            % Allows you to insert figures
\usepackage{amsmath}             % Allows you to do equations
\usepackage{amssymb}
\usepackage{mathtools}           % mathtools builds on top of asmmath
\usepackage{csquotes}
\usepackage{titling}
\usepackage{url}       % For typesetting urls
\usepackage{subcaption}    % For handling subfigures, i.e. nesting figures, list of figures, array of figures, etc

% biblatex is better than default bibtex
\usepackage[
	backend=biber,
	natbib,       % natbib comptability: Create natbib aliases commands that map to biblatex core commands
	style=authoryear,
	autocite=plain]{biblatex}


%%
%% ""Debug"" packages
%%
\usepackage{todonotes}        % For writing todos and notes in yor pdf
\usepackage{lipsum}           % For Lorem Ipsum

\ifbool{DebugBuild} {
	\usepackage{showframe}        % To debug the frame border, padding, etc. It draws a box around each major region of the page
	\usepackage{layout}           % To generate automatically a layout page
} {
	% Else clause
}

%%
%% User level packages, can be customized to fit your need
%%
\usepackage{float}            % use [H] to force figure position exactly as placed in the original latex source
\usepackage{listings}         % For verbatim source code highlighting
\usepackage[linesnumbered, ruled]{algorithm2e}     % To write pseudocode algorithms in a consistent Latex-alike style

\usepackage{tikz}            % A drawing package to draw arbitrary shapes, usefull even for drawing networks/graphs
\usepackage{fancyvrb}        % Better verbatim environment, supporting also centering of the verbatim

\usepackage{setspace}        % For \setstretch


% To access various style for arrow heads
\usetikzlibrary{snakes,arrows,shapes}

\definecolor{webgreen}{rgb}{0,.5,0}
\definecolor{webbrown}{rgb}{.6,0,0}
\definecolor{Pantone}{RGB}{155,0,20}
\definecolor{GrigioLight}{RGB}{152, 152, 152}

\definecolor{codegreen}{RGB}{0,153,0}
\definecolor{codegray}{RGB}{96,96,96}
\definecolor{codered}{RGB}{204,0,0}
\definecolor{codeblue}{RGB}{0,0,153}
\definecolor{codelightblue}{RGB}{0,204,204}
\definecolor{codepurple}{RGB}{0.58,0,0.82}
\definecolor{backcolour}{RGB}{235,235,235}

\hypersetup{
	%hyperfootnotes=false,
	%pdfpagelabels,
	%draft,	% = elimina tutti i link (utile per stampe in bianco e nero)
	colorlinks=true,
	linktocpage=true,
	pdfstartpage=1,
	pdfstartview=FitV,
	% decommenta la riga seguente per avere link in nero (per esempio per la
	%stampa in bianco e nero)
	%colorlinks=false, linktocpage=false, pdfborder={0 0 0}, pdfstartpage=1,
	%pdfstartview=FitV,
	breaklinks=true,
	pdfpagemode=UseNone,
	pageanchor=true,
	pdfpagemode=UseOutlines,
	plainpages=false,
	bookmarksnumbered,
	bookmarksopen=true,
	bookmarksopenlevel=1,
	hypertexnames=true,
	pdfhighlight=/O,
	%nesting=true,
	%frenchlinks,
	urlcolor=webbrown,
	linkcolor=webbrown,
	citecolor=webgreen,
	%pagecolor=RoyalBlue,
	%urlcolor=Black, linkcolor=Black, citecolor=Black, %pagecolor=Black,
	pdftitle={\Title},
	pdfauthor={\textcopyright\ \AuthorName, \UnivName, \DeptName},
	pdfsubject={},
	pdfkeywords={},
	pdfcreator={pdfLaTeX},
	pdfproducer={LaTeX}
}

\captionsetup{
	tableposition=top,
	figureposition=bottom,
	font=small,
	format=hang,
	labelfont=bf
}

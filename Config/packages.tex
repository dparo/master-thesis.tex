%%
%% "Core" packages
%%
\usepackage{ifthen}              % To write sane if-then statemnts
\usepackage{afterpackage}        % To run commands conditionally after a package is loaded (if it is loaded)
\usepackage[htt]{hyphenat}       % To enable hyphenation and provides hyphenatable monospaced fonts, i.e. \texttt{} reflows on new lines correctly
\usepackage{hyperref}            % To allow for interactive links (click with the mouse)
\usepackage{graphicx}            % Allows you to insert figures
\usepackage{amsmath}             % Allows you to do equations
\usepackage{amssymb}
\usepackage{mathtools}           % mathtools builds on top of asmmath
\usepackage{csquotes}
\usepackage{titling}
\usepackage{url}       % For typesetting urls
\usepackage{subcaption}    % For handling subfigures, i.e. nesting figures, list of figures, array of figures, etc

% biblatex is better than default bibtex
\usepackage[style=authoryear,autocite=plain]{biblatex}

%%
%% ""Debug"" packages
%%
\usepackage{todonotes}        % For writing todos and notes in yor pdf
\usepackage{lipsum}           % For Lorem Ipsum

\ifbool{DebugBuild} {
	\usepackage{showframe}        % To debug the frame border, padding, etc. It draws a box around each major region of the page
	\usepackage{layout}           % To generate automatically a layout page
} {
	% Else clause
}



%%
%% User level packages, can be customized to fit your need
%%
\usepackage{float}            % use [H] to force figure position exactly as placed in the original latex source
\usepackage{listings}         % For verbatim source code highlighting
\usepackage[linesnumbered, ruled]{algorithm2e}     % To write pseudocode algorithms in a consistent Latex-alike style

\usepackage{tikz}            % A drawing package to draw arbitrary shapes, usefull even for drawing networks/graphs
\usepackage{fancyvrb}        % Better verbatim environment, supporting also centering of the verbatim

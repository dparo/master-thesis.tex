% Adds comma in in-text citations
\renewcommand*{\nameyeardelim}{\addcomma\space}

\newcommand{\myinputchap}[1]{\input{#1}\clearpage}

\DeclarePairedDelimiter\ceil{\lceil}{\rceil}
\DeclarePairedDelimiter\floor{\lfloor}{\rfloor}


\renewcommand{\(}{\left(}                                     % Automatically scaled open paren
\renewcommand{\)}{\right)}                                    % Automatically scaled close paren
\renewcommand{\[}{\left[}                                     % Automatically scaled open bracket
\renewcommand{\]}{\right]}                                    % Automatically scaled close bracket


% Better parenthesized parenthesis: Use the star version '*' to automatically scale these parenthesis
%%       Example: a \Expr*{ \sum_{i} b_i }
\DeclarePairedDelimiter{\Expr}{\lparen}{\rparen}
\DeclarePairedDelimiter{\Tuple}{\lparen}{\rparen}
\DeclarePairedDelimiter{\Array}{\lbrack}{\rbrack}
\DeclarePairedDelimiter{\Set}{\lbrace}{\rbrace}

% Modulo function
\newcommand{\fmod}[2]{\mathop{mod}\Expr*{#1, #2}}

% Abs function
\DeclarePairedDelimiter{\abs}{\lvert}{\rvert}
\DeclarePairedDelimiter{\SetSize}{\lvert}{\rvert}


\newcommand{\N}{\mathbb N}
\newcommand{\Z}{\mathbb Z}
\newcommand{\Q}{\mathbb Q}
\newcommand{\R}{\mathbb R}



% Shorten \mathrr for equation: this is useful to define multi-character variables in mathemtical equations
\newcommand{\mt}[1]{\mathrm{#1}}

\newcommand{\EqStackTwo}[2]{\substack{#1 \\ #2}}

% Abs function


\newcommand{\mytodo}[1]{\todo[inline]{TODO: #1}}


\newcommand{\lorem}{\lipsum[1]}

% algorithm2e package customization
%   Define \Comment command inside \begin{algorithm}\end{...} environment to define comments
\SetKwProg{Proc}{proc}{}{end}
\SetKwComment{Comment}{/* }{ */}
\SetKw{Assert}{assert}
\SetKwRepeat{Do}{do}{while}          % Define do/while construct

\noindent Il Capacitated Vehicle Routing Problem, abbreivato come CVRP,
è un problema di instradamento di otimizzazione combinatoria  nel quale,
un insieme geograficamente distribuito di clienti con richieste associate note,
devono essere seriviti da una flotta di veicoli stazionati in una struttura centrale.
Nelle ultime due decadi,
tecniche di column generations incorporate all'interno di frameworks branch-price-and-cut
hanno dominato lo stato dell'arte come approccio dominante
per costruire algoritmi esatti per il CVRP.
Il pricer, un imporante componente nella column generation, deve risolvere
il cosidetto Pricing Problem (PP) che richiede la risoluzione di un
Elementary Shortest Path Problem with Resource Constraints (ESPPRC)
in una rete di costo ridotto.
Pochi sforzi scientifici sono stati dedicati allo studi di approacci
branch-and-cut cut per affrontare il PP.
L'ESPRC è stato tradizionalmente rilassato e risolto attraverso algoritmi di programmazione dinamica
Questo approccio, tuttavia, ha due principali svantaggi.
In primo luogo, porta a un peggioramento dei dual bounds ottenuti.
In secondo luogo, il tempo di esecuzione deteriora all'aumentare della lunghezza dei percorsi generati.
La comunità di ricerca operativa ha tradizionalmente impiegato
un insieme di istanze di test storiche e artificiali per valutare
l'esecuzione dei propri contributi.
Purtroppo queste istanze, che sono caratterizzate da piccoli percorsi,
non catturano le proprietà principali dei problemi reali di distribuzione moderna,
in cui i percorsi sono per natura più lunghi.

\noindent In questa tesi sviluppiamo
uno schema basato su un approaccio branch-and-cut per risolvere il pricing problem.
Studiamo il comportamento e l'efficacia dell'approccio proposto in produrre percorsi più lunghi
comparandolo con le soluzioni all'avanguardia che si basano su algoritmi di dinamica.

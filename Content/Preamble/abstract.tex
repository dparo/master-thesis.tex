\noindent The Capacitated Vehicle Routing Problem, CVRP for short,
is a combinatorial optimization routing problem in which,
a geographically distributed set of customers with associated known demands,
must be served with a fleet of vehicles stationed at a central facility.
In the last two decades,
column generation techniques embedded inside branch-price-and-cut frameworks
have been the de facto state-of-the-art dominant approach
for building exact algorithms for the CVRP.
The pricer, an important component in column generation, needs to solve
the so-called Pricing Problem (PP) which asks for an
Elementary Shortest Path Problem with Resource Constraints (ESPPRC)
in a reduced cost network.
Little scientific efforts have been dedicated in studying
branch-and-cut based approaches for tackling the PP.
The ESPPRC has been traditionally relaxed and solved through dynamic programming
algorithms.
This approach, however, has two main downsides.
First, it leads to a worsening of the obtained dual bounds.
Second, the running time deteriorates as the length of the generated paths increases.
The operations research community has traditionally employed
a set of historical and artificial test instances to evaluate
the performance of their contributions.
Unfortunately, these instances, which are characterized by small routes,
they do not capture the main properties of modern real-world distribution problems,
where the routes are longer by nature.

\noindent In this thesis, we develop
a scheme based on a branch-and-cut approach for solving the pricing problem.
We study the behavior and effectiveness of the proposed approach in producing longer paths
by comparing it with state-of-the-art solutions based on dynamic programming algorithms.

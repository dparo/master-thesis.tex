\noindent The Capacitated Vehicle Routing Problem, CVRP for short,
is a combinatorial optimization routing problem in which
a geographically distributed set of
customers with associated known demands must be served with a fleet of vehicles
stationed at a central facility.
In the last two decades,
column generation techniques embedded inside branch-and-price frameworks
have been the de facto state-of-the-art dominant approach
for building exact algorithms for the CVRP.
The pricer, an important component in column generation, needs to solve
the so-called Pricing Problem (PP) which asks for an
Elementary Shortest Path Problem with Resource Constraints (ESPPRC)
in a reduced cost network.
Little scientific efforts have been dedicated in studying
branch-and-cut based approaches for tackling the PP.
In fact, the ESPPRC has been traditionally relaxed and solved through dynamic programming
algorithms which are known to degrade in performance as
the length of the produced routes increases.
A set of historical test instances are usually employed to access
the performance of the scientific contributions.
These historical artificial instances are characterized by small routes
and they do not capture the main properties of modern real-world distribution
problems,
which are instead characterized by much longer routes.

\noindent In this thesis we develop
a branch-and-cut based scheme for solving the pricing problem
and we study its effectiveness and behavior in producing longer routes
by comparing it against the state-of-the-art dynamic programming solution.

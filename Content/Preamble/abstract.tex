{
\setlength\parindent{0pt}

The \textit{Capacitated Vehicle Routing Problem}, CVRP for short, is a combinatorial optimization routing problem in which, a geographically dispersed set of customers with known demands must be served by a fleet of vehicles stationed at a central facility.
\textit{Column generation} techniques embedded within \textit{branch-price-and-cut} frameworks have been the de facto state-of-the-art dominant approach for building exact algorithms for the CVRP over the last two decades.
The \textit{pricer}, a critical component in column generation, must solve the \textit{Pricing Problem} (PP), which asks for an \textit{Elementary Shortest Path Problem with Resource Constraints} (ESPPRC) in a reduced-cost network.
Little scientific efforts have been dedicated to studying branch-and-cut based approaches for tackling the PP.

The ESPPRC has been traditionally relaxed and solved through dynamic programming algorithms.
This approach, however, has two major drawbacks.
For starters, it worsens the obtained dual bounds.
Furthermore, the running time degrades as the length of the generated routes increases.
To evaluate the performance of their contributions, the operations research community has traditionally used a set of historical and artificial test instances.
However, these benchmark instances do not capture the key characteristics of modern real-world distribution problems, which are usually characterized by longer routes.

In this thesis, we develop a scheme based on a branch-and-cut approach for solving the pricing problem.
We study the behavior and effectiveness of our implementation in producing longer routes by comparing it with state-of-the-art solutions based on dynamic programming.
Our results suggest that branch-and-cut approaches may supplement the traditional labeling algorithm, indicating that further research in this area may bring benefits to CVRP solvers.
}

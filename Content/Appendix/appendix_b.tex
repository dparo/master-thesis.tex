\chapter{Additional details}

\section{A simple dual formulation}\label{sec:cptp-simple-dual-formulation}

A simple dual formulation for the ESPPRC is presented in \cite{beasley1989algorithm} and it is here presented and re-adapted for the CPTP problem.

We first need to re-formulate the CPTP as an ESPPRC over a directed network.
We define a new drected network formulation $G_d = \Tuple*{V_d, E_d}$ composed of $N + 1$ nodes, where the depot node is essentially split in two nodes: a source and a sink vertex.
In the new network we use the value $0$ to denote the source version of the depot node, and $N$ its duplicate playing the role of the sink vertex.
More formally, $V_d = \Set*{ 0,\dots, N }$, and the edge set $E_d$ is

\begin{equation}
	\begin{aligned}
		E_d = & \Set*{\Tuple*{i, j} \quad \forall i, j \in V_d,\ i \ne j } \\
		      & \setminus \Set*{ \Tuple*{0, N} }                           \\
		      & \setminus \Set*{ \Tuple*{i, 0} \quad \forall i \in V_d }   \\
		      & \setminus \Set*{ \Tuple*{N, i} \quad \forall i \in V_d }   \\
	\end{aligned}
\end{equation}

which briefly means: all possible pairs $\Tuple*{i, j},\ i \ne j$ excluding: direct connection between $0$ and $N$, and by exluding edges entering node $0$ and flowing out of node $N$.
At each directed edge $\Tuple*{i, j} \in E_d$ we associate a cost $c_{ij} \in \R$.
We define a new set of binary variables $e_{ij} \in \Set*{0, 1} \quad \forall \Tuple*{i, j} \in E_d$, to model solutions for ESPPRC over the directed network.

Is is easy to see that $2 y_i$ in the original CPTP formulation can now be rewritten in the ESPPRC formulation as:

\begin{equation}
	2 y_i = \ExprESPPIngoingEdges{i} + \ExprESPPOutgoingEdges{i} \quad \forall i \in V_0
\end{equation}

Since we are creating a new node, we are also duplicating some edges and we must pay attention that these edges bind to the same CPTP original formulation.
We can bind the new edges to the original CPTP formulation by employing a mutually exclusive constraint of the form:

\begin{equation}
	e_{0i} + e_{iN} \le 1 \quad \forall i \in V_0
\end{equation}

Finally, by setting $c_{ij} = \ExprCptpReducedCost{i}{j} \quad \forall \Tuple*{i, j} \in E_d$, we can rewrite the CPTP original formulation in the ESPPRC IP model:

\begin{align}
	\min_{e} \quad z_\mt{ESPPRC}(e) & =  \sum_{(i, j) \in E_d} \Expr*{ \ExprCptpReducedCost{i}{j} } e_{ij} \label{eq:espprc-obj-function}                                                                                            \\
	                                & B \le \sum_{i \in V_0} \frac{q_i}{2} \Expr*{ \ExprESPPIngoingEdges{i} + \ExprESPPOutgoingEdges{i} }  \le Q                            \label{eq:espprc-resource-upper-bound-constraint}        \\
	                                & \ExprESPPIngoingEdges{i} = \ExprESPPOutgoingEdges{i}                                                \qquad \forall i \in V_0          \label{eq:espprc-flow-conservation-constraint-customers} \\
	                                & \ExprESPPOutgoingEdges[i]{0} = 1                                                                                                      \label{eq:espprc-flow-conservation-constraint-depot1}    \\
	                                & \ExprESPPIngoingEdges[i]{N} = 1                                                                                                       \label{eq:espprc-flow-conservation-constraint-depot2}    \\
	                                & \ExprESPPEdgesWithin[S] \le |S| - 1                                                                  \qquad \forall S \subseteq V_d   \label{eq:espprc-sec-constraints}                        \\
	                                & e_{0i} + e_{iN} \le 1                                                                                \qquad \forall i \in V_0         \label{eq:espprc-cptp-binding}                           \\
	                                & e_{ij}                   \in \lbrace 0, 1 \rbrace                                                    \qquad \forall \Tuple*{i, j} \in E_d    \label{eq:espprc-e-mip-var-bounds}                \\
\end{align}

where, \eqref{eq:espprc-obj-function} is the new objective function where we dropped the dependency on the $y$ MIP variable,
\eqref{eq:espprc-resource-upper-bound-constraint} is the resource bound expressed as a function of only the $e$ MIP variable,
\eqref{eq:espprc-flow-conservation-constraint-customers} is the flow conservation for all vertices except the source and sink node
and \eqref{eq:espprc-flow-conservation-constraint-depot1}, \eqref{eq:espprc-flow-conservation-constraint-depot2} are the flow conservation for respectively the source and sink node,
\eqref{eq:espprc-cptp-binding} avoids forming paths where a single customer is visited,
and finally \eqref{eq:espprc-sec-constraints} are the new subtour elimination constraints which avoid the forming of spurious subtours in unconnected regions of nodes.

\cite{beasley1989algorithm} propose to ignore the SEC in \eqref{eq:espprc-sec-constraints} and relaxing \eqref{eq:espprc-resource-upper-bound-constraint}, \eqref{eq:espprc-cptp-binding} in a Lagrangean fashion:

\begin{align}
	\min_{e} \quad z_\mt{ESPP}(e) & = \lambda_0 B - \lambda_1 Q + \sum_{(i, j) \in E_d}  c'_{ij} e_{ij} - \sum_{i \in V_0} \beta_i  \label{eq:esp-relaxed-obj-function} \\
	                              & \ExprESPPIngoingEdges{i} = \ExprESPPOutgoingEdges{i}        \qquad \forall i \in V_0                                                \\
	                              & \ExprESPPOutgoingEdges[i]{0} = 1                                                                                                    \\
	                              & \ExprESPPIngoingEdges[i]{N} = 1                                                                                                     \\
	                              & e_{ij}                   \in \lbrace 0, 1 \rbrace           \qquad \forall \Tuple*{i, j} \in E_d
\end{align}

where:

\begin{equation}
	c'_{ij} =
	\begin{cases}
		\Expr*{\ExprCptpReducedCost{i}{j}} + \Expr*{\lambda_1 - \lambda_0} \frac{q_i + q_j}{2},           & \texttt{if } i \in V_0,\ j \in V_0 \\
		\Expr*{\ExprCptpReducedCost{i}{j}} + \Expr*{\lambda_1 - \lambda_0} \frac{q_i + q_j}{2} + \beta_j, & \texttt{if } i = 0,\ j \in V_0     \\
		\Expr*{\ExprCptpReducedCost{i}{j}} + \Expr*{\lambda_1 - \lambda_0} \frac{q_i + q_j}{2} + \beta_i, & \texttt{if } i \in V_0,\ j = N
	\end{cases}
\end{equation}

and $\lambda_0 \ge 0, \lambda_1 \ge 0, \beta_i \ge 0 \quad \forall i \in V_0$ are the respective Lagrange multipliers.
The above IP describes an elementary shortest path problem, where the associated weights may be negative.

\cite{beasley1989algorithm} propose a subgradient procedure to obtain feasible lower bound for the original CPTP problem.
Since the ESPP problem with arbitrary weights is an NP-hard problem, they suggest to use a standard Dijkstra algorithm, but in order to do so, the lagrange multipliers must be set such that $c'_{ij} \ge 0$ is satisfied.
By considering only lagrange multipliers achieving $c'_{ij} \ge 0$ has the additional benefit that neglecting the SEC were justified since these would be automatically satisfied.
The advantage of this approach is that it allows for the usage of simple of the shelf algorithms to obtain lower bounds for the CPTP.
But, unfortunately, the lower bounds generated by such approach are unsatisfactory at best.
As pointed out in \cite{righini2004dynamic}, solving the ESPPRC problem through Lagrangean relaxation is effective only when the Lagrangean subproblem is easy, i.e. for which $c'_{ij} \ge 0$ over a good portion of the lagrange multipliers search space.

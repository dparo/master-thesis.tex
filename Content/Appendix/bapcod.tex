\chapter{\bapcod{} Parametrization}
\label{sec:bapcod-appendix}

This section exhaustively lists all of the configuration parameters
for respectively the \bapcod{} and \vrpsolver{} extension.
These parameters were employed to obtain the results in \cref{sec:results}.
We will not provide an exhaustive list of each parameter;
instead, we will focus on the significant ones employed to achieve the desired intent.
The \bapcod{} technical report \parencite{sadykov2021}
provides a more comprehensive list and description of the available parameters.
The technical report alone is not enough to comprehend the purpose of each parameter,
therefore it is highly suggested to follow the scientific document of \textcite{pessoa2020generic}.

\medskip

The parameters we've used will be summarized in the following sections of this appendix,
along with their associated values, meaning, and justification for their use.
First, in the remainder of this section,
we want to provide a summary of the guiding decisions
that influenced the choice of the parameters.

Non-robust inequalities (branching and cutting planes)
are necessarily disabled since our pricer does not support them.
While robust branching and cut-generation are technically doable,
we chose to disable these two features entirely to simplify the BAP algorithm
and reduce the computation time.
Evaluating the column generation iterations issued at the root node
of the branch-and-bound tree without additional cutting planes or branching
is more than sufficient to stress the labeling algorithm.

We want to apply as much pressure on the \vrpsolver{}'s pricer as possible.
The \textit{ng-sets augmentation} \parencite{pessoa2020generic} is enabled,
and the ng-sets maximum set size is raised as high as possible.
Furthermore, the \vrpsolver{} pricer's tailing-off condition was disabled.
The labeling algorithm will work harder to produce dual bounds closer to the elementary bound
as the ng-sets extend.
As a result, the dual bound at the root node will improve significantly.

The \vrpsolver{} extension's labeling algorithm can generate multiple routes
per pricing iteration, providing the BPC with more diverse paths.
We disabled such a feature to produce a fair comparison because our BAC framework
can only output a single column per pricing iteration.

\section{Misc parameters}

These parameters control numerous aspects of \bapcod{}, such as the logging verbosity.
These aren't especially significant parameters,
but they can help with debugging/understanding what's going on inside the BPC algorithm's guts.

\begin{itemize}
	\item \texttt{DEFAULTPRINTLEVEL = 0}.
	      Controls the verbosity of the \bapcod{}'s logging system by setting it to a low level.
	      This value strikes an acceptable balance between the amount
	      of information emitted and the number of characters printed per second.
	\item \texttt{printMasterPrimalSols = 2}.
	      By setting this parameter to 2, \bapcod{} will print the fractional solution after each column generation convergence.
	      Useful for debugging.
\end{itemize}

\section{Core parameters}

These parameters govern the BPC algorithm's overall behavior,
such as total running time, which pricer to use, enabling specific components
and performing validation checks.

\begin{itemize}
	\item \texttt{GlobalTimeLimit = 3600}.
	      The maximum allowed solving time is set to one hour.
	      This global timelimit affects the entire BPC algorithm
	      from the beginning to the resolution process at the root node
	      until the discovery of an integral optimal solution.
	\item \texttt{colGenSubProbSolMode = 3}.
	      Tells \bapcod{} to use the user-supplied custom pricing functor.
	      In our case, the pricing functor is a stubbed implementation that wraps
	      the original \vrpsolver{} extension pricing functor
	      and times it
	      (see previous discussions in \Cref{sec:results-evaluation-process}).
	      Instead, by setting this parameter to the value \texttt{2},
	      we can instruct \bapcod{} to use the default generic MIP pricer for solving the pricing sub-problems.
	\item \texttt{ApplyPreprocessing = true}.
	      Tells \bapcod{} to utilize pre-processing to adjust bounds and remove redundant constraints.
	      We enabled this feature because it can potentially improve the model
	      without affecting the pricing sub-problem.
	\item \texttt{PreprocessVariablesLocalBounds = false}.
	      Ensures that \bapcod{} does not change the variable bounds of sub-problems after pre-processing.
	      We disable this feature because our pricer does not support modifying variables' bounds.
	\item \texttt{CheckSpOracleFeasibility = false}.
	      This feature is helpful for debugging and when developing custom pricing functors.
	      When this parameter is set to \texttt{true},
	      an expensive MIP resolution process validates the pricer algorithm's correctness.
	      The MIP determines whether the generated columns satisfy all the MP's constraints.
	      It is disabled in our case.
	\item \texttt{CheckOracleOptimality = false}.
	      Similar to the \texttt{CheckSpOracleFeasibility} parameter but it also verifies the optimality of the generated columns.
\end{itemize}

\section{Column Generation parameters}

These parameters control the column generation framework.

\begin{itemize}
	\item \texttt{GenerateProperColumns = false}.
	      When this parameter is set to \texttt{true},
	      \bapcod{} will throw an error if the pricer produces a non-proper column,
	      that is, a column that violates the sub-problem variable bounds.
	      In our case, we set this parameter to \texttt{false}
	      because it did not appear to cooperate with the \vrpsolver{} pricing functor in our tests.
	\item \texttt{MaxNbOfStagesInColGenProcedure = 3}.
	      Sets the number of stages that the pricer may use.
	      When the stage iteration is greater than one,
	      the pricer is permitted to use progressively lighter heuristics for column determination.
	      The pricing problem should be solved exactly when the stage iteration reaches stage zero.
	      When the previous stage converges,
	      namely when it ails to find any reduced cost column,
	      \bapcod{} automatically lowers the current stage number.
	      When using the \vrpsolver{} extension,
	      the technical documentation recommends setting this parameter to \texttt{3}.
	      The \vrpsolver{} extension implements two heuristic stages using
	      a modified bidirectional labeling algorithm.
	      See the \textit{Pricing heuristics} and \textit{Column and cut generation} sections of \textcite{sadykov2021bucket}.
	\item \texttt{ReducedCostFixingThreshold = 0.0}.
	      Setting it to zero disables reduced cost fixing.
\end{itemize}

\section{Cut Generation parameters}
These parameters govern the generation of cutting planes.
We've decided to turn off cut generation entirely in the BPC framework.

\begin{itemize}
	\item \texttt{MasterCuttingPlanesDepthLimit = -1}.
	      When this parameter is set to \texttt{-1}, core cut generation is disabled.
	      Core cuts are problem-independent cutting planes generated directly from the BPC algorithm.
	      Because this parameter does not affect user-defined cuts, they must be disabled manually.
	      Explicit usage of user-defined cuts is requested explicitly
	      by registering the associated functors.
	      Consequentially in our \bapcod{} model, we do not register any additional user-defined cuts.
	      As a result, all cutting planes,
	      including the separation of the RCC inequalities \cref{eq:two-index-flow-ccc},
	      are disabled.
	\item \texttt{MaxNbOfCutGeneratedAtEachIter = 0}.
	      Sets the maximum number of core cuts added per cut round to zero,
	      effectively enforcing the core cuts generation's deactivation.
\end{itemize}

\section{Branching parameters}
These parameters govern the BPC algorithm's branching scheme and branching priorities.
We've decided to turn off all forms of branching.

\begin{itemize}
	\item \texttt{StrongBranchingPhaseOne=""}, \texttt{StrongBranchingPhaseTwo=""}, \texttt{StrongBranchingPhaseThree=""}, \texttt{StrongBranchingPhaseFour=""}
	      These parameters govern the strong branching behavior of \bapcod{},
	      comprised of several phases.
	      We disable each phase's strong branching
	      by setting the corresponding parameter to an empty string.
	\item \texttt{SimplifiedStrongBranchingParameterisation = false}.
	      When this parameter is set to true, \bapcod{} populates the parameters
	      \texttt{StrongBranchingPhaseOne}, \texttt{StrongBranchingPhaseTwo}, \texttt{StrongBranchingPhaseThree}, \texttt{StrongBranchingPhaseFour}
	      with sane default values.
	      We set \texttt{SimplifiedStrongBranchingParameterisation} to \texttt{false}
	      because populating these aforementioned parameters would effectively re-enable branching.
\end{itemize}

\section{\vrpsolver{} extension parameters}

These settings affect the labeling algorithm, the size of the ng-sets, the route enumeration procedure
and a few other minor aspects of the \vrpsolver{} extension.
Recall that the \vrpsolver{} extension was introduced in \textcite{pessoa2020generic},
where it quickly became one of the most advanced strategies for solving routing-like problems.
Refer to the scientific paper of \textcite{pessoa2020generic} for additional details.

\begin{itemize}
	\item \texttt{RCSPuseBidirectionalSearch = 2}.
	      This parameter, known as $\phi^\mt{bidir}$ in the scientific paper,
	      instructs the RCSP pricer to use bidirectional search to solve the pricing problem.
	\item \texttt{RCSPapplyReducedCostFixing = 1}.
	      This parameter, known as $\phi^\mt{elim}$ in the scientific paper,
	      instructs the pricer to use the standard bucket arc elimination
	      discussed in \textcite{sadykov2021bucket}.
	\item \texttt{RCSPmaxNumOfColsPerExactIteration = 1}, \texttt{RCSPmaxNumOfColsPerIteration = 1}.
	      These parameters, known as $\gamma^\mt{exact}, \gamma^\mt{heur}$  in the scientific paper,
	      control the number of generated columns per iteration for the heuristic and exact stages, respectively.
	      Because our pricer can only output a single column per pricing iteration,
	      we force the labeling algorithm to operate similarly to ensure a fair comparison.
	\item \texttt{RCSPallowRoutesWithSameVerticesSet = false}.
	      This parameter tells the labeling algorithm whether it is permissible
	      to output multiple routes that pass through the same vertices.
	      When disabled, it allows the pricer to generate more diversified paths
	      at the expense of increasing the pricing run time.
	      This parameter has no effect because we're forcing the labeling algorithm
	      to output a single route regardless.
	\item \texttt{RCSPmaxNumOfEnumSolutionsForMIP = 1}.
	      This parameter, known as $omegamtMIP$ in the scientific paper,
	      controls the \textit{route enumeration} \parencite{baldacci2008} threshold,
	      at which the number of enumerated paths
	      can directly trigger a solution of the SP formulation via a MIP optimizer.
	      Because we want to measure pricing complexity,
	      we disable the route enumeration feature by setting the corresponding parameter to \texttt{1}.
	\item \texttt{RCSPstopCutGenTimeThresholdInPricing = 1e21}, \texttt{RCSPhardTimeThresholdInPricing = 1e21}.
	      These parameters, known as $\tau^\mt{soft}$ and  $\tau^\mt{hard}$ in the scientific paper,
	      affect the soft and hard time thresholds.
	      These thresholds, applicable only during the exact pricing stage,
	      modify the tailing-off condition of the pricer.
	      If the pricer's running time exceeds one of these thresholds,
	      column generation is preemptively interrupted in favor of cut generation or branching.
	      We want to measure the label setting algorithm's performance even when it struggles,
	      so we set those parameters to high values to disable the tailing-off condition.
	\item \texttt{RCSPdynamicNGmode = 1}.
	      This parameter tells \bapcod{} to scale dynamically
	      the ng-sets size based on the
	      fractional solution obtained by the column generation.
	      Note that ng-set augmentation \textbf{is not} based on the solution received from the pricing problem.
	      See \textcite{pessoa2020generic} for more details.
	\item \texttt{RCSPinitNGneighbourhoodSize = 8}, \texttt{RCSPmaxNGneighbourhoodSize = 63}.
	      These parameters, known as $\eta^\mt{init}$ and $\eta^\mt{max}$ in the scientific paper,
	      limit the the ng-set size during augmentation.
	      Raising $\eta^\mt{init}$ can improve the dual bound generated by the pricer,
	      but it usually results in an explosion of computation times.
	      As a result, if one wants to improve the dual bound at each branch-and-bound node,
	      keeping $\eta^\mt{init}$ small and increasing $\eta^\mt{max}$ is the preferred approach.
	      The ng-sets size starts at the lower value $\eta^\mt{init}$ and it is later increased (if \texttt{RCSPdynamicNGmode} is enabled)
	      after the column generation convergence and only if the MP contains a fractional solution (see \cite{pessoa2020generic}).
	      Since our pricer produces the best dual-bound improvement possible (elementary routes),
	      we want to stress the labeling algorithm to try as hard as possible to achieve similar results.
	      Therefore, by setting $\eta^\mt{init} = 8$ and $\eta^\mt{max} = 63$
	      we force the labeling algorithm to produce better dual bounds towards the end of the column generation convergence.
	      Note that, due to \bapcod{}'s implementation details $\eta^\mt{max} \le 63$ and thus
	      elementary routes can be guaranteed in the column generation
	      only for instances having less than $64$ nodes.
	\item \texttt{RCSPmaxNGaverNeighbourhoodSize = 63}.
	      This parameter is very similar to the $\eta^\mt{max}$ parameter, but for the average case.
	      In our case we set this parameter to the same value of $\eta^\mt{max}$
	      as it was suggested from Ruslan Sadykov,
	      a researcher who worked on \bapcod{} and \vrpsolver{} extension implementations.
	\item \texttt{CutTailingOffThreshold = 0.0000001}, \texttt{CutTailingOffCounterThreshold = 999999}.
	      These parameters, known as $\delta^\mt{gap}$ and $\delta^\mt{num}$ in the scientific paper,
	      control the tailing off condition of cut-separation;
	      namely, the threshold at which the BPC deems cuts ineffective
	      and switches to lower priority (and usually more computationally expensive) cuts.
	      Despite their name and application to cut separation,
	      these two critical parameters also affect the column generation procedure:
	      the augmentation of the ng-sets.
	      As stated in the original paper, increasing the ng-sets size is a form of "cut generation"
	      because it improves the dual bound. Ng-sets augmentation has the highest priority,
	      followed by cut-generation and branching in this order.
	      We want to disable the cut-tailing off condition
	      because we always prefer to augment the ng-sets as much as possible.
	      Setting these parameters to the above values allows us to disable the cut-tailing off condition.
\end{itemize}

\begin{comment}
[ About what is reduced cost fixing ]
[ Bucket arc elimination procedure = Reduced cost fixing procedure]
\textcite{sadykov2021}
VRPSolver extension includes an implementation of the pricing functor which
allows the user to define the subproblems as resource constrained shortest path
problems in graphs. The functor implements the bucket-graph based labeling
algorithm from paper [16] for solving the pricing problem, as well as the corre-
sponding bucket arc elimination procedure (i.e. reduced cost fixing procedure),
and the elementary route enumeration procedure [1]. VRPSolver extension also
implements cut separation functors for rounded cap
\end{comment}

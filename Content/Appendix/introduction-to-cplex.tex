\chapter{Introduction to CPLEX}
\label{sec:introduction-to-cplex}

\urlref{https://www.ibm.com/analytics/cplex-optimizer}{IBM ILOG CPLEX Optimizer},
CPLEX for short,
is a commercial optimization software package for solving problems expressed as either:
linear programs, mixed-integer programs, quadratic programs, or quadratically constrained programs.
ILOG CPLEX was purchased by IBM in 2009 and is still maintained by IBM.
CPLEX is available in two flavors: a standalone executable and a callable library.
The executable can solve problems defined interactively or defined inside a custom file format.
The CPLEX Callable library is a C dynamic library linkable with custom user code.
CPLEX also includes language bindings for C++, Java, Python, and other popular programming languages.
CPLEX uses the simplex algorithm (both primal and dual)
and an advanced proprietary algorithm based on a branch-and-cut technique called \textit{Dynamic Search}.
CPLEX applies pre-processing at the root to improve the formulation and reduce the problem size.
It also features a probing technique for analyzing logical implications by setting binary variables to $0$ or $1$.
It applies several heuristics at each node of the tree: diving heuristics, Local Branching (LB) \parencite{fischetti2003}, Feasibility pump \parencite{fischetti2005}, Relaxation Induced Neighborhood \parencite{danna2005}, Evolutionary genetic algorithms for polishing the solutions \parencite{rothberg2007}, and more.
CPLEX includes efficient implementations of several cut templates.
Among the most notable: Gomory cuts \parencite{chvatal1973}, Knapsack covers \parencite{letchford2020lifting}, GUB covers \parencite{wolsey1990valid}, Local Branching \parencite{fischetti2003}, Flow covers \parencite{padberg1985valid}, Clique covers \parencite{brigham1983clique}, 0-1 half cuts \parencite{caprara1996}, and more.
Moreover, the node-selection and branching decision strategies in CPLEX are top-notch.
The node-selection and branching decision strategies
are the most delicate component of a branch-and-cut algorithm
because they can substantially alter the whole running time of the procedure \parencite{lodi2013performance}.
They received many years of development and tuning to make them top-notch.
For more details about CPLEX refer to \textcite{lima2010, lodi2013heuristic}.

MIP solvers are rather general and can be used to solve a wide range of problems from various fields \parencite{bixby2007progress}.
MIP models are, in spirit, a way to mathematically program a solver to achieve the desired solution.
A MIP solver can solve a mixed-integer linear programming formulation
expressed as \parencite{wolsey1999integer}:
\begin{align}
	 & \max_{x, y} & c^T x + d^T y                                 \\
	 & \text{s.t.} & A x + B y \le b  \label{eq:general-mip-bound} \\
	 &             & x \in \R^n                                    \\
	 &             & y \in \Z_{+}^k,
\end{align}
where $A \in \R^{m \times n}, B \in \R^{m \times k}$ are matrices and
$c \in \R^n, d \in R^k, b \in \R^m$ are vector coefficients.
The bound in \cref{eq:general-mip-bound} can also be rewritten in equality and/or greater form.

MIP programs are solvable in various ways,
but the most common is to use the simplex algorithm (for solving the continuous relaxation)
and a branch-and-cut algorithm (to handle the integrality constraints).
The branch-and-cut algorithm,
first introduced in \textcite{padberg1991},
is a hybridization between a cutting plane method
and a branch-and-bound procedure \parencite{land2010}.
The branch-and-bound algorithm is a divide-and-conquer algorithm
that manipulates a search tree to explore the solution space.
Each node of the tree represents a sub-problem of the feasible region.
The continuous relaxation was proven to be solvable in polynomial time in \textcite {khachiyan1979polynomial}.
Even though the simplex algorithm works well in practice,
it could rarely take an exponential amount of time to complete.
Regardless, the simplex algorithm works flawlessly in practice.

MIP solvers are exact by nature: given enough running time,
they can compute a proven optimal solution by exhibiting both a lower and upper bound
to the objective function.
They tend to be less effective than an ad-hoc heuristic algorithms
since they work with a generic abstract mathematical model
and cannot anticipate all the nuances that the model itself represents.
As \textcite{bixby1999} points out,
MIP modeling is a powerful and convenient technique that wasn't practical 40 years ago due
to limited hardware and inadequate implementations.
MIP models are now quite manageable and solvable in a reasonable time on consumer hardware,
thanks to numerous reasons:
advances in processor speed, the development of faster algorithms,
embedded heuristics, better preprocessing, post-processing, and polishing techniques.

Unfortunately, MIP solvers are frequently quite complex.
Lots of efforts are needed to develop an efficient MIP solver.
Unlike a simple heuristic whose implementation spans a few lines of code,
a MIP solver cannot be developed in "house" using limited resources within a reasonable time.
The most efficient MIP solvers on the market have licenses that cost tens of thousands of dollars.

Fortunately, CPLEX is available freely through an academic license at this
\urlref{https://www.ibm.com/support/knowledgecenter/SSSA5P\_12.5.1/ilog.odms.cplex.help/CPLEX/GettingStarted/topics/preface/preface\_synopsis.html}{URL}.

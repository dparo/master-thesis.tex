\chapter{Introduction to CPLEX}
\label{sec:introduction-to-cplex}

\urlref{https://www.ibm.com/analytics/cplex-optimizer}{IBM ILOG CPLEX Optimizer},
CPLEX for short,
is a commercial optimization software package for solving problems expressed as either:
linear programs, mixed-integer programs, quadratic programs, or quadratic constrained programs.
ILOG CPLEX was bought from IBM in 2009 and is currently maintained by IBM.
CPLEX comes in different flavors as either an independent executable or as a callable library.
The executable can solve problems defined interactively or defined inside a custom file format.
The CPLEX Callable library is a C dynamic library linkable with custom user code.
CPLEX also comes with bindings for different languages such as C++, Java, Python, and more.
CPLEX uses the simplex algorithm (both primal and dual)
and an advanced proprietary algorithm based on a branch-and-cut technique called \textit{Dynamic Search}.
CPLEX applies pre-processing at the root to improve the formulation and reduce the problem size.
It also features a probing technique to analyze logical implications by fixing binary variables to either $0$ or $1$.
It applies several heuristics at each node: Diving heuristic, Local Branching \parencite{fischetti2003}, Feasibility pump \parencite{fischetti2005}, Relaxation Induced Neighborhood \parencite{danna2005}, Evolutionary genetic algorithms for polishing the solutions \parencite{rothberg2007}, and more.
CPLEX includes efficient implementations of several cut templates.
Among the most notable: Gomory cuts \parencite{chvatal1973}, Knapsack covers \parencite{letchford2020lifting}, GUB covers \parencite{wolsey1990valid}, Local Branching \parencite{fischetti2003}, Flow covers \parencite{padberg1985valid}, Clique covers \parencite{brigham1983clique}, 0-1 half cuts \parencite{caprara1996}, and more.
Moreover, the node-selection and branching decision strategies in CPLEX are top notch.
The node-selection and branching decision strategies
are the most delicate component of a branch-and-cut algorithm
because they can substantially alter the whole running time of the procedure \parencite{lodi2013a}.
They received many years of development and tuning to make them top-notch.
For more details about CPLEX refer to \textcite{lima2010, lodi2013}.

MIP solvers are quite general
and can be employed to solve a variety of problems from different fields \parencite{bixby2007progress}.
MIP models are, in spirit, a way to mathematically program a solver to achieve the desired solution.
A MIP solver can solve a mixed integer linear programming formulation
expressed as \parencite{wolsey1999integer}:

\begin{align}
	 & \max_{x, y} & c^T x + d^T y                                 \\
	 & \text{s.t.} & A x + B y \le b  \label{eq:general-mip-bound} \\
	 &             & x \in \R^n                                    \\
	 &             & y \in \Z_{+}^k
\end{align}

where $A \in \R^{m \times n}, B \in \R^{m \times k}$ are matrices and
$c \in \R^n, d \in R^k, b \in \R^m$ are vector coefficients.
The bound in \cref{eq:general-mip-bound} can also be rewritten in equality and/or greater form.

MIP programs are solvable in various ways,
but the most typical one is to use the simplex algorithm
(for solving the continuous relaxation)
and to employ a branch-and-cut algorithm (to handle the integrality constraints).
The branch-and-cut algorithm,
first introduced in \textcite{padberg1991},
is a hybridization between a cutting plane method
and a branch-and-bound procedure \parencite{land2010}.
Branch-and-bound is a divide-and-conquer algorithm
that builds a search tree to explore the solution space.
Each node of the tree represents a sub-problem of the feasible region.
The continuous relaxation is solvable in polynomial time,
even though the simplex algorithm,
which works well in practice, can, in the worst case, take exponential running time.

MIP solvers are exact by nature: given enough running time, they can compute a proven optimal solution by exhibiting both a lower bound and an upper bound to the objective function.
They tend to be less effective than an ad-hoc heuristic algorithm since they work with a generic abstract mathematical model and cannot foresee all the nuances that the model itself represents.
As \textcite{bixby1999} says, MIP modeling is a powerful and convenient technique but wasn't very applicable 40 or so years ago due to limited hardware and less efficient implementations.
Thanks to advances in processor speed, the development of faster and smarter algorithms, embedded heuristics, preprocessing, postprocessing, and polishing techniques, MIP models are nowadays quite manageable and solvable in a reasonable time on consumer hardware.

Unfortunately, MIP solvers are usually quite complicated.
Lots of research and development is spent on creating an efficient MIP solver.
Unlike a simple heuristic which implementation spans a few lines of code, a MIP solver cannot be developed in "house" using limited resources within a reasonable time.
Most efficient MIP solvers on the market have licenses costing in the range of tens of thousands of dollars.

Fortunately, CPLEX is available freely through an academic license at this
\urlref{https://www.ibm.com/support/knowledgecenter/SSSA5P\_12.5.1/ilog.odms.cplex.help/CPLEX/GettingStarted/topics/preface/preface\_synopsis.html}{URL}.

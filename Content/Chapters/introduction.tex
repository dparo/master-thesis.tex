\chapter{Introduction}
\label{sec:introduction-chapter}

\section{The Capacitated Vehicle Routing Problem}
\label{sec:intro-cvrp-problem}
The \textit{Capacitated Vehicle Routing Problem} (\textbf{CVRP}), first presented in \textcite{dantzig1959}
under the name of "truck dispatching problem",
is one of the most studied combinatorial optimization routing problems.
The CVRP is an NP-hard (in the strong sense) problem
that can be considered a generalization of the well-known Travelling Salesman Problem (TSP).
The TSP \parencite{flood1956}
is an NP-hard \parencite{garey1976planar} ubiquitous combinatorial optimization problem in the operations research field,
that asks for the determination of a Hamiltonian circuit of minimum cost
\parencite{croes1958, laporte1992,johnson1997,applegate2006,gutin2006,hoffman2013}.
The CVRP can be defined verbally as finding an optimal route for a transportation/distribution/delivery problem
starting from a common point called the depot,
where a homogeneous fleet composed of a fixed number of trucks, subject to capacity constraints,
need to serve customer demands of a single good (i.e. delivery of gasoline to gas stations).
Given as input: a weighted graph representing the road network,
the customer demands and the vehicle capacity,
the problem consists in determining a set of non-oriented routes, one for each vehicle,
of minimal overall travel distance starting and ending at the depot.
The set of routes needs to serve all the customers in the road network exactly once
while satisfying the vehicle capacity bound \parencite{toth2014}.
A diagram showing an example of a CVRP problem along with its optimal solution
is provided in \Cref{fig:cvrp-optimal-solution-example}.

\begin{figure}[t]
	\centering
	\includegraphics[width=12cm]{Imgs/P-n40-k5-solution.out.cropped.pdf}
	\caption{An illustration of an optimal CVRP solution (instance named P-n40-k5).
		There are $39$ customers to serve and $5$ trucks with a capacity of $140$ each.
		Each color in the picture represents a different route that each truck has taken.
		Instead, the colored dots represent the customers served by each truck on its route.
		The customer demands are not depicted for the sake of clarity.
		The larger black dot in the center represents the depot where each route must begin and end.
		Credits: \url{http://vrp.galgos.inf.puc-rio.br/index.php/en/plotted-instances?data=P-n40-k5}.
	}
	\label{fig:cvrp-optimal-solution-example}
\end{figure}

Investigating effective CVRP solution methods may result in
significant real-world economic savings for the management
of the provision of goods or services in a distribution system.
Optimal delivery planning can reduce the overall transportation, goods costs,
and waiting time experienced by the customers.
As a result, researching efficient exact algorithms and mathematical models
for solving and describing real-world distribution problems
becomes critical for the operational management
of a cost-effective planning process \parencite{toth2002,toth2014}.

The CVRP is part of a larger class of problems known as the Vehicle Routing Problems (VRPs).
There are many variations of VRPs proposed in the literature, including
the \textit{Vehicle Routing Problem with Time Windows} (VRPTW) \parencite{schrage1981}
and many others.
The vehicles in the VRPTW are subject to capacity constraints
and must serve each customer within an allocated time window slot.
Nonetheless, CVRP is the simplest VRP variant to describe,
and to this day, it remains the most central and studied routing problem.
For a comprehensive taxonomy of the many VRP variants, refer to \textcite{eksioglu2009, braekers2016}.

While effective (meta-)heuristic algorithms have been proposed and applied
successfully to many VRP variants to obtain good-enough solutions
in reduced computation time,
the focus of this thesis is on exact algorithms for solving the CVRP.

We can find major contributions employing heuristics for the VRP, among others, in
\textcite{clarke1964, desrochers1989matching, paessens1988savings, foster1976integer}.
Meta-heuristics approaches for the VRP can be found in
\textcite{gendreau1994tabu, cordeau2012parallel, toth2003granular, li2005very, pisinger2007, kytojoki2007efficient, nagata2009,vidal2012, subramanian2013},
just to name a few.

For a more comprehensive survey on (meta-)heuristics for the VRP, refer to
\textcite{golden1998impact,gendreau2002metaheuristics,gendreau2008,laporte2014chapter,elshaer2020taxonomic}.

Exact algorithms are typically slower than (meta-)heuristics, but given
enough computation time, they can produce a proven optimal solution.
They accomplish this by closing the objective function's primal-dual bound gap.

\medskip

We strongly recommend the book  "\citetitle{toth2014}" of \textcite{toth2014}
for a comprehensive overview/survey on the CVRP and VRPTW problems,
as well as other common VRP variants.
This book served as a good reference and was instrumental in laying the groundwork
for the first chapters of this thesis.
We also gathered additional information from other VRP surveys in the works of
\textcite{cordeau2007, baldacci2012, caceres-cruz2015, costa2019}.

\section{Thesis Contributions}
\label{sec:intro-thesis-contributions}

\mytodo{This section must be rewritten entirely at the face of the latest radical changes to the thesis.}

One of the major problems of contemporary CVRP solvers
is that they are usually developed and tuned through the usage
of historical benchmark instances which
have little or no similarities with real-world distribution problems.
The major historical test instances,
employed to assess the performance of the scientific contributions,
have been classified in different sets, or families.
Each family is identified through a single upper case letter.
We here summarize the core sets proposed by the operations research community for the CVRP problem:
(i) set E is proposed in \textcite{dantzig1959, christofides1969, gaskell1967bases, gillett1974heuristic}
where some instances are randomly generated whereas for some others
the authors don't provide description on how they were generated,
(ii) set M is proposed in \textcite{christofides1979vehicle} and
it's obtained by aggregating together instances from the E set,
(iii) set F is proposed in \textcite{fisher1994} and it's obtained from an actual distribution problem of groceries in the city of Ontario,
finally (iv) sets A, B, P are proposed in \textcite{augerat1995} and are generated artificially: A is random, B is clustered, P generated from A, B, E by changing capacities.
Another less common test set is the one proposed in \textcite{golden1998impact},
which contains large scale instances ranging from 200 to 480 customers,
generated programmatically following concentric geometric figures.

These benchmark instances have become quite easy for modern algorithms.
They suffer from being either too homogeneous or too artificial,
while not covering the main characteristics found in current real-world distribution problems.
Despite some efforts in proposing a complete, diverse and more contemporary common denominator
of set instances for the CVRP (set X, proposed in \textcite{uchoa2017}),
the common and historical instances played (and still remains) the main central test-bed for comparing
and assessing the performance of CVRP contributions.

These historical instances are usually characterized with stringent vehicle capacities
which consequently give rise to optimal solutions characterized by small routes, each visiting few customers.
In practice, the vehicle capacity is rarely the bottleneck and
real-world modern distribution problems are instead characterized by much longer routes.
The dynamic programming labeling algorithm
\parencite{desrochers1992,feillet2004}
is the commonly employed method to tackle the Pricing Problem (PP) \parencite{gutierrez-jarpa2010, archetti2011, bettinelli2011, contardo2014, contardo2015, pecin2017new, pecin2017improved, pessoa2020generic}.
It has been shown that the computational performance of such algorithm
tends to deteriorate as the length of the generated routes increases.

The goal of the thesis lies in studying the feasibility and competitiveness of a
branch-and-cut algorithm, implemented through a commercial MIP software package,
for solving the pricing problem.
The objective is to empirically evaluate the performance of the proposed
branch-and-cut framework against the state-of-the-art labeling
algorithm, while empirically measuring how the two frameworks behave
as the lengths of the route they need to generate increases.
\cite{jepsen2014} proposed a branch-and-cut framework for solving the pricing problem.
In their work, \citeauthor{jepsen2014} showed that, despite the dynamic programming algorithm
turned out to be much faster in many instances, the BAC framework exhibited
better performance for some larger instances; proving that
a BAC framework could complement the dynamic programming algorithm for solving the PP.
This thesis stems from the original work \citeauthor{jepsen2014}.
The objective of this thesis is to revisit the work of \citeauthor{jepsen2014},
and verify whether recent development improvements in MIP optimizers
have made them competitive at solving the pricing problem,
or vice versa,
if the current situation and modern dynamic programming algorithms have
made BAC approaches completely obsolete.

\section{Outline}
\label{sec:intro-outline}

This section provides an overview of the work's contents.
The thesis' contents can be divided into three major portions, which we will list here.

The first portion provides introductory technical material
regarding the CVRP and contemporary resolution methods for this problem.
\Cref{sec:cvrp-mathematical-formulations} formalizes the CVRP
through multiple mathematically rigorous integer programming models.
\Cref{sec:intro-literature-review} summarizes the notable contributions
of the operations research practitioners regarding CVRP's exact approaches.
The branch-and-price (BPC) algorithm, column generation,
and the pricing sub-problem are presented in \cref{sec:branch-and-price}.
The most recent exact algorithms for solving routing problems are all BPC-based.
Recent research has empirically demonstrated that BPC algorithms
perform admirably in the routing problem domain.

In the second part of the thesis,
we will go over the pricing sub-problem in greater detail,
as well as our branch-and-cut (BAC) pricer implementation.
The pricing sub-problem is a combinatorial optimization problem
that arises when a CVRP is solved using a BPC approach.
\Cref{sec:the-pricing-problem} introduces and discusses the pricing sub-problem
through multiple mathematically rigorous integer programming models.
The implementation details of our proposed branch-and-cut pricer
are presented in \cref{sec:implementation-chapter}.

The third part of the thesis evaluates the competitiveness of the proposed BAC framework
in solving the pricing sub-problem.
We evaluate its performance and compare it to modern cutting-edge solutions.
In \cref{sec:results} we present the evaluation setup, we list the instances employed,
and finally, we show the empirical results along with a discussion.
Finally, in \cref{sec:conclusions} we summarize the work, review the results
and offer helpful tips while discussing potential implementation improvements.

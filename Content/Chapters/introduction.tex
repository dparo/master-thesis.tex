\chapter{Introduction}

\section{The Capacitated Vehicle Routing Problem (CVRP)}

The Capacitated Vehicle Routing Problem (CVRP), first formulated in \cite{dantzig1959truck},
can be considered as a generalization of
the well-known Travelling Salesman Problem (TSP).
The problem consists in finding an optimal route for a delivery problem,
where a fleet composed of a fixed number of trucks, subject to capacity constraints,
need to serve customer demands.
Given as input a weighted graph, the customer demands and the vehicle capacity,
the problem consists in minimizing the total travel distance of all the routes,
while serving all the customers.

\subsection{Mathematical formulations}

\subsubsection{Compact formulation}

\subsubsection{Set Partitioning (SP) formulation}

\section{Resolution methods}

\section{Column generation and the Pricing Problem}

\section{Literature Review}

\section{Thesis contributions}
The objective of the thesis is to study the feasibility of employing a MIP solver
based on a Branch \& Cut Framework to solve the pricing problem induced in solving
CVRP instances.
A similar study was already performed in \cite{Jepsen2014}.
Part of the objective of the thesis is to revisit the work of Jepsen comparing
the performance of a Branch \& Cut Framework against the state of the art work
\cite{pessoa2020generic}.
One of the core contributions is the verification of whether a Branch \& Cut Framework
can perform better than the dynamic labeling algorithm employed \cite{pessoa2020generic},
especially when the optimal routes tend to be longer than usual.

One of the main issues in the work of \cite{pessoa2020generic}, is that
most of the algorithms were optimized for dated datasets, the problem
is that these instances are quite anachronistic, in practice, in modern
distribution problems, the vehicle capacity is rarely the bottleneck.
The same truck within a single day is meant to serve a higher number of customers
compared to the widely employed standard datasets.
It is known that the labeling algorithm tend to deteriorate in performance when
the optimal routes that need to be generated tend to become longer.

This thesis aims at providing a first proof-of-concept answer to the question
whether it is better to employ a Branch \& Cut framework to solve the pricing problem
when the routes become longer.

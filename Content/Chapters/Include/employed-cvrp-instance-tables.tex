\begin{table*}[thb]
	\centering
	% Place caption here to get a caption above the table
	\begin{tabular}[t]{lccc}
		\toprule
		\textbf{Instance} & $K$ & $Q$ & \textbf{Optimal Value} \\
		\midrule
		E-n51-k5          & 5   & 160 & 521                    \\
		E-n76-k7          & 7   & 220 & 682                    \\
		E-n76-k8          & 8   & 180 & 735                    \\
		E-n76-k10         & 10  & 140 & 830                    \\
		E-n76-k14         & 14  & 100 & 1021                   \\
		E-n101-k8         & 8   & 200 & 815                    \\
		E-n101-k14        & 14  & 112 & 1067                   \\
		\bottomrule
	\end{tabular}
	\caption{Instances of the set E employed for the empirical evaluation.
		These instances were originally proposed in \textcite{dantzig1959, christofides1969, gaskell1967bases, gillett1974heuristic}
		where the node locations were generated at random from a uniform distribution \parencite{uchoa2017}.
	}
	\label{table:cvrp-instance-family-E}
\end{table*}

\begin{table*}[thb]
	\centering
	% Place caption here to get a caption above the table
	\begin{tabular}[t]{lccc}
		\toprule
		\textbf{Instance} & $K$ & $Q$   & \textbf{Optimal Value} \\
		\midrule
		F-n45-k4          & 4   & 2010  & 724                    \\
		F-n72-k4          & 4   & 30000 & 237                    \\
		F-n135-k7         & 7   & 2210  & 1162                   \\
		\bottomrule
	\end{tabular}
	\caption{Instances of the set F employed for the empirical evaluation.
		These instances were originally proposed in \textcite{fisher1994},
		and they come from an actual distribution problem involving grocery deliveries in Ontario \parencite{uchoa2017}.
	}
	\label{table:cvrp-instance-family-F}
\end{table*}

\begin{table*}[thb]
	\centering
	% Place caption here to get a caption above the table
	\begin{tabular}[t]{lccc}
		\toprule
		\textbf{Instance} & $K$ & $Q$ & \textbf{Optimal Value} \\
		\midrule
		A-n37-k5          & 5   & 100 & 669                    \\
		A-n37-k6          & 6   & 100 & 949                    \\
		A-n38-k5          & 5   & 100 & 730                    \\
		A-n39-k5          & 5   & 100 & 822                    \\
		A-n39-k6          & 6   & 100 & 831                    \\
		A-n44-k6          & 6   & 100 & 937                    \\
		A-n45-k6          & 6   & 100 & 944                    \\
		A-n45-k7          & 7   & 100 & 1146                   \\
		A-n46-k7          & 7   & 100 & 914                    \\
		A-n48-k7          & 7   & 100 & 1073                   \\
		A-n53-k7          & 7   & 100 & 1010                   \\
		A-n54-k7          & 7   & 100 & 1167                   \\
		A-n55-k9          & 9   & 100 & 1073                   \\
		A-n60-k9          & 9   & 100 & 1354                   \\
		A-n61-k9          & 9   & 100 & 1034                   \\
		A-n62-k8          & 8   & 100 & 1288                   \\
		A-n63-k9          & 9   & 100 & 1616                   \\
		A-n64-k9          & 9   & 100 & 1401                   \\
		A-n65-k9          & 9   & 100 & 1174                   \\
		A-n69-k9          & 9   & 100 & 1159                   \\
		A-n80-k10         & 10  & 100 & 1763                   \\
		\bottomrule
	\end{tabular}
	\caption{Instances of the set A employed for the empirical evaluation.
		These instances were originally proposed in \textcite{augerat1995}
		where the node locations are generated at random on a square grid \parencite{uchoa2017}.
	}
	\label{table:cvrp-instance-family-A}
\end{table*}

\begin{table*}[thb]
	\centering
	% Place caption here to get a caption above the table
	\begin{tabular}[t]{lccc}
		\toprule
		\textbf{Instance} & $K$ & $Q$ & \textbf{Optimal Value} \\
		\midrule
		B-n38-k6          & 6   & 100 & 805                    \\
		B-n39-k5          & 5   & 100 & 549                    \\
		B-n41-k6          & 6   & 100 & 829                    \\
		B-n43-k6          & 6   & 100 & 742                    \\
		B-n44-k7          & 7   & 100 & 909                    \\
		B-n45-k6          & 6   & 100 & 678                    \\
		B-n50-k7          & 7   & 100 & 741                    \\
		B-n50-k8          & 8   & 100 & 1312                   \\
		B-n51-k7          & 7   & 100 & 1032                   \\
		B-n52-k7          & 7   & 100 & 747                    \\
		B-n56-k7          & 7   & 100 & 707                    \\
		B-n57-k7          & 7   & 100 & 1153                   \\
		B-n57-k9          & 9   & 100 & 1598                   \\
		B-n63-k10         & 10  & 100 & 1496                   \\
		B-n64-k9          & 9   & 100 & 861                    \\
		B-n66-k9          & 9   & 100 & 1316                   \\
		B-n67-k10         & 10  & 100 & 1032                   \\
		B-n68-k9          & 9   & 100 & 1272                   \\
		B-n78-k10         & 10  & 100 & 1221                   \\
		\bottomrule
	\end{tabular}
	\caption{Instances of the set B employed for the empirical evaluation.
		These instances were originally proposed in \textcite{augerat1995}
		where the node locations are generated at random on a square grid \parencite{uchoa2017}.
	}
	\label{table:cvrp-instance-family-B}
\end{table*}

\begin{table*}[thb]
	\centering
	% Place caption here to get a caption above the table
	\begin{tabular}[t]{lccc}
		\toprule
		\textbf{Instance} & $K$ & $Q$  & \textbf{Optimal Value} \\
		\midrule
		P-n16-k8          & 8   & 35   & 450                    \\
		P-n19-k2          & 2   & 160  & 212                    \\
		P-n20-k2          & 2   & 160  & 216                    \\
		P-n21-k2          & 2   & 160  & 211                    \\
		P-n22-k2          & 2   & 160  & 216                    \\
		P-n22-k8          & 8   & 3000 & 603                    \\
		P-n23-k8          & 8   & 40   & 529                    \\
		P-n40-k5          & 5   & 140  & 458                    \\
		P-n45-k5          & 5   & 150  & 510                    \\
		P-n50-k7          & 7   & 150  & 554                    \\
		P-n50-k8          & 8   & 120  & 631                    \\
		P-n50-k10         & 10  & 100  & 696                    \\
		P-n51-k10         & 10  & 80   & 741                    \\
		P-n55-k7          & 7   & 170  & 568                    \\
		P-n55-k8          & 8   & 160  & 588                    \\
		P-n55-k10         & 10  & 115  & 694                    \\
		P-n55-k15         & 15  & 70   & 989                    \\
		P-n60-k10         & 10  & 120  & 744                    \\
		P-n60-k15         & 15  & 80   & 968                    \\
		P-n65-k10         & 10  & 130  & 792                    \\
		P-n70-k10         & 10  & 135  & 827                    \\
		P-n76-k4          & 4   & 350  & 593                    \\
		P-n76-k5          & 5   & 280  & 627                    \\
		P-n101-k4         & 4   & 400  & 681                    \\
		\bottomrule
	\end{tabular}
	\caption{Instances of the set P employed for the empirical evaluation.
		These instances were originally proposed in \textcite{augerat1995}
		obtained from modifying the capacities of some instances of the A, B and E sets \parencite{uchoa2017}.
	}
	\label{table:cvrp-instance-family-P}
\end{table*}

\chapter{IP mathematical formulation}

Let $G = \left(V, E \right)$ denote a complete undirected graph, where $V = \lbrace 0, 1, \dots, N - 1 \rbrace$ denotes the set of nodes,
$E = \lbrace  e = (i, j) \mid i,j \in V, j \ge i + 1 \rbrace$ the set of edges, and $N$ the number of nodes in the graph.
The value $0 \in V$ is used to denote the depot node.

Let $p_{i} \in \R, p_{i} \ge 0$ denote the profit function, and $q_{i} \in \R, q_{i} \ge 0$ denote the demand function, which represent respectively the gain and required demand in visiting a vertex $i \in V$,
Note that without loss of generality we can allow for $p_{0} > 0$ and $d_{0} > 0$.
Let $c_{ij} \in \R, c_{ij} > 0$ denote the distance function between a pair of nodes  $i, j \in V$.
We assume that the distance function satisfies the triangle inequality:

\begin{equation}
	c_{ij} \le c_{ih} + c_{hj}
\end{equation}



Declare the variables here.

Given an upper bound $Q \in \R$ for the total resource consumption, we can finally express the CPTP problem as an integer programming formulation:


\begin{align}
	\min_{x,y} \quad z(x, y) & =  \sum_{i \in V} \sum_{\substack{j \in V                                                                                \\ j \ge i + 1}} c_{ij} x_{ij} - \sum_{i \in V} p_i y_i\\
	                         & y_0 = 1                                           & \label{eq:depot-part-of-tour-constraint}                             \\
	                         & \sum_{\substack{j \in V                                                                                                  \\ j \ge i + 1}}       x_{ij}    = 2 \quad \forall i \in V         \label{eq:flow-conservation-constraint}\\
	                         & \sum_{(i, j) \in \delta(S)} x_{ij} \ge 2 y_{i}    & \quad \substack{\forall i \in S                                      \\ \forall S \subseteq V \setminus \{ 0 \},\ |S| \ge 2} \label{eq:gsec-constraints} \\
	                         & \sum_{i \in V} d_i y_i   \le Q                    & \label{eq:resource-upper-bound-constraint}                           \\
	                         & x_{ij}                   \in \lbrace 0, 1 \rbrace & \quad \forall (i, j) \in E               \label{eq:x-mip-var-bounds} \\
	                         & y_{i}                    \in \lbrace 0, 1 \rbrace & \quad \forall i \in V,\ i \ge 1          \label{eq:y-mip-var-bounds} \\
\end{align}





\subsection{Valid inequalities}

\subsubsection{GLM inequalities}

\subsubsection{RCI inequalities}


\section{Separation technique of cuts and valid inequalities}

\subsection{GSEC separation}

\subsection{GLM separation}

\subsection{RCI separation}

\section{Warm start}

\chapter{IP formulation}

\cite{Jepsen2014} provided in their work a functional IP formulation for the CPTP.
Most of the model in this section is derived from their work, and it is here summarized.

Let $G = \left(V, E \right)$ denote a complete undirected graph, where $V = \lbrace 0, 1, \dots, N - 1 \rbrace$ denotes the set of nodes,
$E = \lbrace  e = (i, j) \mid i,j \in V, j \ge i + 1 \rbrace$ the set of edges, and $N$ the number of nodes in the graph.
The value $0 \in V$ is used to denote the depot node.
For convenience, we define $V_0 = V \setminus \{0\}$ to express the set of customers, and $N_0 = N - 1$ to denote the number of customers.
Let $\delta(S)$ with $S \subseteq V$ denote the edges crossing the set $S$ and its complement $\overline{S} = V \setminus S$.
More formally we can express $\delta(S)$ as $\delta(S) = \lbrace (i, j) \in E \mid i \in \lparen S \cap V \rparen, j \in \lparen \overline{S} \cap V \rparen \rbrace$.
For brevity, we also define $\delta(i) = \delta(\{i\})$ to denote the set of edges incident to node $i \in V$.

Let $p_{i} \in \R, p_{i} \ge 0$ denote the profit function, and $q_{i} \in \R, q_{i} \ge 0$ denote the demand function, which represent respectively the profit gain and required demand in visiting a vertex $i \in V$.
By convention we define $p_0 = 0$ and $q_0 = 0$.
Let $d_{ij} \in \R, d_{ij} > 0$ denote the distance function between a pair of nodes  $i, j \in V$.
We assume that the distance function is symmetric $d_{ij} = d_{ji}$ and satisfies the triangle inequality $d_{ij} \le d_{ih} + d_{hj}$.

We define two sets of binary variables: $x_{ij}$, $y_{i} \in \lbrace 0, 1 \rbrace$, used to compose output solutions for the CPTP.
Together they model respectively an edge and a vertex as being part of the tour solution.

Given an upper bound $Q \in \R,Q \ge 0$ for the total resource consumption, we can finally express the CPTP problem as an integer programming formulation


\begin{align}
	\min_{x,y} \quad z(x, y) & =  \sum_{i \in V} \sum_{\substack{j \in V                                                                                 \\ j \ge i + 1}} d_{ij} x_{ij} - \sum_{i \in V} p_i y_i \label{eq:obj-function}\\
	                         & y_0 = 1                                           & \label{eq:depot-part-of-tour-constraint}                              \\
	                         & \sum_{j \in \delta(i)}       x_{ij}    = 2        & \quad \forall i \in V         \label{eq:flow-conservation-constraint} \\
	                         & \sum_{(i, j) \in \delta(S)} x_{ij} \ge 2 y_{i}    & \quad \substack{\forall i \in S                                       \\ \forall S \subseteq V_0,\ |S| \ge 2} \label{eq:gsec-constraints} \\
	                         & \sum_{i \in V} q_i y_i   \le Q                    & \label{eq:resource-upper-bound-constraint}                            \\
	                         & x_{ij}                   \in \lbrace 0, 1 \rbrace & \quad \forall (i, j) \in E               \label{eq:x-mip-var-bounds}  \\
	                         & y_{i}                    \in \lbrace 0, 1 \rbrace & \quad \forall i \in V,\ i \ge 1          \label{eq:y-mip-var-bounds}
\end{align}

The equation \eqref{eq:obj-function} denotes the objective function, and it is made of two terms: the cost of traveling through the tour (positive contribution) and the total sum of the profits associated to visiting a subset of the vertices (negative contribution).
The constraint \eqref{eq:depot-part-of-tour-constraint} ensures that the depot is always visited.
Constraints \eqref{eq:flow-conservation-constraint} imposes flow conservation constraints for each vertex.
The equation \eqref{eq:gsec-constraints} are the generalized subtour elimination constraints \textbf{(GSECs)} and must be present to guarantee that the solution does not contain spurious unconnected subtours.
There is an exponential number of GSECs, as such, for performance reasons these constraints cannot be inserted statically in a MIP solver, and must instead be efficiently separated dynamically.
The equation \eqref{eq:resource-upper-bound-constraint} models the fact that a tour may visit any vertex as long as the total demand doesn't exceed the vehicle capacity.
Finally, the equations \eqref{eq:x-mip-var-bounds}, \eqref{eq:y-mip-var-bounds} specify the bounds and domain of the $x_{ij}$ and $y_{i}$ variables respectively.

The IP model consists of $\frac{N^2 + N}{2}$ number of binary variables.
If we ignore the GSECs and the binary variable bounds, the total number of constraints characterizing the model add up to $N + 2$.

Note that the IP formulation only permits solutions visiting at least three vertices (the depot and two customers).
We could allow for single customer solutions by relaxing the binary variable requirements and by modifying the IP model.
Instead, it may be simpler to take a different approach and leave the IP model as is.
By employing a bruteforce algorithm in $\Theta(N)$ time we can scan for improving single-customer solutions.
After the resolution of the IP model, we check for improving single-customer solutions over the solution returned from the IP model.
If any of such solutions exist, we update the incumbent.

\subsection{GSECs inequalities}\label{sec:gsec-inequality}

\mytodo{Add more details/history/literature about GSECs here.}


It may be interesting to study the relationship between the number of non-zero binary variables participating in a GSEC as a function of the set size $|S|$.
To achieve this we can express the size of the edge crossing set $\delta(S)$ as

\begin{equation}\label{eq:delta-s-set-size}
	|\delta(S)| = |S| (N - |S|)
\end{equation}

which is maximized when $|S| = \frac{N}{2}$, thus leading to the following easy upper bound:

\begin{equation}\label{eq:delta-s-set-size-ub}
	|\delta(S)| \le \frac{N^2}{4}
\end{equation}

Let $\mt{NNZ}_{\mt{GSEC}}(S)$ denote the number of non-zero binary variables participating in a GSEC inequality.
It is easy to see that the following result holds
\begin{equation}\label{eq:gsec-nnz-ub}
	\mt{NNZ}_{\mt{GSEC}}(S) \le 1 + \floor*{\frac{N^2}{4}} \quad \forall S \subseteq V_0,\ |S| \ge 2
\end{equation}
This trivial bound allows us to preallocate a fixed amount of memory beforehand, saving us copious memory allocation time that would otherwise be necessary at each GSEC separation routine call.

\subsection{Additional bounds}

\subsubsection{Demand lower bound}\label{sec:demand-lower-bound}
Although non strictly-necessary, it is possible to improve the linear relaxation of the IP model by introducing a static constraint modeling a lower bound on the minimum served demand:

\begin{equation}\label{eq:resource-lower-bound-constraint}
	\sum_{i \in V} q_i y_i   \ge B
\end{equation}

where $B \in \R, B \ge 0$ is a constant and represents the the required minimum served demand.
The constant $B$ value can be computed as $B = q_u + q_v$, where $u, v \in V_0$ represent the least two demanding customers among all the customers, i.e. $q_u \le q_v \le q_i \quad \forall i \in V_0,\ i \ne u, v$.

\section{A simple dual formulation}

A simple dual formulation for the ESPPRC is presented in \cite{beasley1989algorithm} and it is here presented and re-adapted for the CPTP problem:

\mytodo{Write about the method here}


\section{Additional Valid Inequalities}\label{sec:additional-valid-inequalities}

In this section we present additional valid inequalities for the CPTP problem.
These inequalities are not strictly required to solve a CPTP, but when embedded inside a Branch and Cut framework can heavily speed up the resolution process.
Most of these cuts follow the presentation proposed in \cite{Jepsen2014}.
We will concentrate mostly on the most important cuts, additional cuts and information are treated in more detail in \cite{Jepsen2014}.

\subsection{Generalized Large Multistar (GLM) inequalities}
The Generalized Large Multistar inequalities, GLM for short, were first proposed in \cite{gouveia_result_1995}, and further discussed in \cite{letchford2006projection}.
In the additional work of \cite{letchford_multistars_2002}, the GLM cuts are further generalized in the so-called Knapsack Large Multistar (KLM) inequalities.
The GLM inequalities can be expressed as:

\begin{equation}\label{eq:glm-inequality-v1}
	\begin{split}
		\sum_{(i, j) \in \delta(S)} x_{ij} \ge \frac{2}{Q} \left(  \sum_{i \in S} q_i y_i + \sum_{i \in S} \sum_{j \in V_0 \setminus S} q_j  x_{ij}\right) \quad \forall S \subseteq V_0,\ |S| \ge 2
	\end{split}
\end{equation}

It is important to recall that we defined $q_0 = 0$.
Therefore $\sum_{i \in S} \sum_{j \in V_0 \setminus S} q_j  x_{ij}$ becomes

\begin{equation}
	\sum_{i \in S} \sum_{j \in V_0 \setminus S} q_j  x_{ij} = \sum_{i \in S} \sum_{j \in V \setminus S} q_j  x_{ij} = \sum_{(i, j) \in \delta(S)} q_j x_{ij}
\end{equation}

Finally, equation \eqref{eq:glm-inequality-v1} can be rewritten more concisely as:

\begin{equation}\label{eq:glm-inequality}
	\begin{split}
		\sum_{(i, j) \in \delta(S)} x_{ij} \left( 1 - 2 \frac{q_j}{Q} \right)  -2 \sum_{i \in S} y_i \frac{q_i}{Q}  \ge  0   \quad \forall S \subseteq V_0,\ |S| \ge 2
	\end{split}
\end{equation}


Let $\mt{NNZ}_{\mt{GLM}}(S)$ denote the number of non-zero binary variables participating in a GLM inequality as a function of the set $S$.
By using equation \eqref{eq:glm-inequality} and the result obtained in \eqref{eq:delta-s-set-size}, we have

\begin{equation}
	\mt{NNZ}_{\mt{GLM}}(S) = -|S|^2 + (N + 1)|S|
\end{equation}

which is maximized when $|S| = \frac{N+1}{2}$.
It is now easy to see that the following result holds
\begin{equation}\label{eq:glm-nnz-ub}
	\mt{NNZ}_{\mt{GLM}}(S) \le \floor*{ \frac{\left( N + 1 \right)^2}{4}} \quad \forall S \subseteq V_0,\ |S| \ge 2
\end{equation}


\subsection{Rounded Capacity Inequalities (RCI)}
The Rounded Capacity Inequalities, RCI for short, were first presented in \cite{achuthan_capacitated_1998}.

Let $q(S) = \sum_{i \in S} q_i$ and $Q_{R}(S) = \mathop{mod}\left(q(S), Q \right)$ be respectively the total demand and remainder capacity associated to set $S \subseteq V_0$.
The RCI inequalities can then be expressed as:

\begin{equation}
	\begin{split}
		\sum_{(i, j) \in \delta(S)} x_{ij} -2 \sum_{i \in S} y_i {\frac{q_i}{Q_\mt{R}(S)}}    \ge   2 \left( \ceil*{ \frac{q(S)}{Q}} - \frac{q(S)}{Q_{\mathrm{R}}(S)} \right) \quad \forall S \subseteq V_0
	\end{split}
\end{equation}

Let $\mt{NNZ}_{\mt{RCI}}(S)$ denote the number of non-zero binary variables participating in a RCI inequality as a function of the set $S$.
The upper bound on $\mt{NNZ}_{\mt{RCI}}(S)$ follows the same reasoning of equation \eqref{eq:glm-nnz-ub}:

\begin{equation}\label{eq:rc-nnz-ub}
	\mt{NNZ}_{\mt{RCI}}(S) \le \floor*{\frac{\left( N + 1\right)^2}{4}} \quad \forall S \subseteq V_0
\end{equation}

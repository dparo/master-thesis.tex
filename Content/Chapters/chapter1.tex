\chapter{IP mathematical formulation}

\cite{Jepsen2014} provided in their work a functional IP formulation for the CPTP.
Most of the model in this section is derived from their work, and it is here summarized.

Let $G = \left(V, E \right)$ denote a complete undirected graph, where $V = \lbrace 0, 1, \dots, N - 1 \rbrace$ denotes the set of nodes,
$E = \lbrace  e = (i, j) \mid i,j \in V, j \ge i + 1 \rbrace$ the set of edges, and $N$ the number of nodes in the graph.
The value $0 \in V$ is used to denote the depot node.
Let $\delta(S)$ with $S \subseteq V$ denote the edges crossing the set $S$ and its complement $\overline{S}$.
More formally we can express $\delta(S)$ as $\delta(S) = \lbrace (i, j) \in E \mid i \in \lparen S \cap V \rparen, j \in \lparen \overline{S} \cap V \rparen \rbrace$.

Let $p_{i} \in \R, p_{i} \ge 0$ denote the profit function, and $q_{i} \in \R, q_{i} \ge 0$ denote the demand function, which represent respectively the gain and required demand in visiting a vertex $i \in V$.
Note that without loss of generality we can allow for $p_{0} > 0$ and $d_{0} > 0$.
Let $c_{ij} \in \R, c_{ij} > 0$ denote the distance function between a pair of nodes  $i, j \in V$.
We assume that the distance function satisfies the triangle inequality:

\begin{equation}
	c_{ij} \le c_{ih} + c_{hj}
\end{equation}


We define two sets of binary variables: $x_{ij}$, $y_{i} \in \lbrace 0, 1 \rbrace$, used to compose output solutions for the CPTP.
Together they model respectively an edge and a vertex as being part of the tour solution.

Given an upper bound $Q \in \R,Q \ge 0$ for the total resource consumption, we can finally express the CPTP problem as an integer programming formulation


\begin{align}
	\min_{x,y} \quad z(x, y) & =  \sum_{i \in V} \sum_{\substack{j \in V                                                                                \\ j \ge i + 1}} c_{ij} x_{ij} - \sum_{i \in V} p_i y_i \label{eq:obj-function}\\
	                         & y_0 = 1                                           & \label{eq:depot-part-of-tour-constraint}                             \\
	                         & \sum_{\substack{j \in V                                                                                                  \\ j \ge i + 1}}       x_{ij}    = 2 \quad \forall i \in V         \label{eq:flow-conservation-constraint}\\
	                         & \sum_{(i, j) \in \delta(S)} x_{ij} \ge 2 y_{i}    & \quad \substack{\forall i \in S                                      \\ \forall S \subseteq V \setminus \{ 0 \},\ |S| \ge 2} \label{eq:gsec-constraints} \\
	                         & \sum_{i \in V} d_i y_i   \le Q                    & \label{eq:resource-upper-bound-constraint}                           \\
	                         & x_{ij}                   \in \lbrace 0, 1 \rbrace & \quad \forall (i, j) \in E               \label{eq:x-mip-var-bounds} \\
	                         & y_{i}                    \in \lbrace 0, 1 \rbrace & \quad \forall i \in V,\ i \ge 1          \label{eq:y-mip-var-bounds} \\
\end{align}.


The equation \eqref{eq:obj-function} denotes the objective function, and it represents the cost of traveling through the tour (positive contribution) and the total sum of the profits associated to visiting a subset of the vertices (negative contribution).
The constraint \eqref{eq:depot-part-of-tour-constraint} ensures that the depot is always visited.
Constraints \eqref{eq:flow-conservation-constraint} imposes flow conservation constraints for each vertex.
Notice that the flow entering each node can be equal to either $0$ or $1$ depending on the vertex is visited in the tour.
The equation \eqref{eq:gsec-constraints} are the generalized subtour elimination constraints \textbf{(GSECs)} and must be present to guarantee that the solution does not contain spurious unconnected subtours.
There is an exponential number of GSECs, as such, for performance reasons these constraints cannot be inserted statically in a MIP solver, and must instead be efficiently separated dynamically.
The equation \eqref{eq:resource-upper-bound-constraint} models the fact that a tour may visit any vertex as long as the total demand does surpass the vehicle capacity.
Finally, the equations \eqref{eq:x-mip-var-bounds}, \eqref{eq:y-mip-var-bounds} specify the bounds and domain of the $x_{ij}$ and $y_{i}$ variables respectively.

The IP model consists of $\frac{N^2 + N}{2}$ number of binary variables.
If we ignore the GSECs and the binary variable bounds, the total number of constraints characterizing the model add up to $N + 2$.

Note that the IP formulation allows for solutions visiting at least three vertices (the depot plus two customers).
We could allow for single customer solutions by relaxing the binary variable requirements and by modifying slightly the IP model.
Instead, it may be simpler to take a different approach without modifying the IP model.
In fact, given an optimal solution generated from the IP model, by using a bruteforce algorithm in $\Theta(N)$ time scan for improving solutions visiting single customers.
If such feasible solutions exist, we update the incumbent.



\section{Valid Inequalities}

\section{Separation techniques}

\subsection{GSEC separation}

\subsection{GLM separation}

\subsection{RCI separation}

\section{Improving the formulation}

\section{Warm start}

\chapter{Results}
\label{sec:results}

\section{\bapcod}
\label{sec:results-bapcod}

\textit{\bapcod}\ \parencite{sadykov2021} is a software package
developed in France at the Bordeaux University and Bordeaux Research Center ,
which implements a modern branch-cut-and-price algorithm
and a column generation procedure.
\bapcod\ takes as input a mixed integer programming model, and solves it by applying
a Dantzig-Wolfe reformulation \parencite{dantzig1960}.
The branch-and-price framework is generic and may be used standalone.
In fact, it supports user developed extensions,
which can provide custom cutting-planes, branching-decisions
or user-defined pricers.
The \textit{VRPSolver} extension \parencite{pessoa2020a}, is
a \bapcod\ extension distributed by the same authors.
This extension implements an RCSP pricer, some cutting-planes
and branching decisions to efficiently solve routing-like problems,
such as the CVRP, VRPTW, but also other problems that exhibit similar
structures (see \cite{pessoa2020a}).
For more detail about \bapcod and its VRPSolver extension refer to
\textcite{sadykov2021}.

\bapcod\ along with its VRPSolver are currently the leading state-of-the-art
technology for solving vehicle routing problems \parencite{pessoa2020a}.




\section{CVRP Benchmark Instances}
\label{sec:results-benchmark-instances}

\mytodo{Introduce the original benchmark used and their naming convention, and filename convention <F>-n<X>-k<Y>}

\mytodo{Talk about how benchmark instances are obtained}

\mytodo{Say that they follow the TSPLIB95 file format}

\mytodo{Say that the distances are rounded. Refer to \cite{uchoa2017}, which contains a small explanation why this is done}

\begin{comment}
\cite{uchoa2017}
THE CVRPLIB website
The typical instance repository of today is a web page that allows downloading the instance
files and includes additional textual information, like file format description, instance
source, best known/optimal solution values, etc. The CVRLIB web page, where the new
instances (and all the previous CVRP instances described in Section 2) are available
(http://vrp.galgos.inf.puc-rio.br/index.php/en/), is more sophisticated:
\end{comment}

\mytodo{Put all the instances used in tables as is done in \cite{uchoa2017}}


\section{Evaluation Method}
\label{sec:results-evaluation-method}



\section{Performance profiles}
\label{sec:results-performance-profiles}

\textit{Performance profiles} are a plot representation used to benchmark optimization software,
first presented in \textcite{dolan2002}.
Performance profiles are simple to interpret and less subject to personal interpretation.
In short, performance profiles plot the cumulative distribution function with respect to a performance metric.
We test $H$ algorithms by running them on $M$ problem instances.
In the X-axis, we plot the performance metric ratio with respect to a baseline.
The baseline is computed as the best performance achieved by all the algorithms under analysis.
The Y-axis instead shows the probability of being within an X ratio from the baseline.

\

We've extensively employed performance profiles to compare the competitiveness of
our proposed branch-and-cut based pricer
against the RCSP dynamic programming algorithm used by \bapcod\ \parencite{pessoa2020a}.

In this thesis, we will use performance profiles extensively to measure each solver by exploiting two performance metrics: \textbf{Time metric}, \textbf{Cost metric}.

A \textbf{Time performance profile} will tell us which resolution method is the fastest in terms of runtime.
A \textbf{Cost performance profile}, instead, will show us the cost ratio of the best upper bound obtained from each resolution method.
The ground truth optimal, as extracted from the dataset, is used instead as the cost baseline.





\section{Empirical Results}
\label{sec:results-empirical-results}
\mytodo{Spam this section with performance profiles}

\section{Discussion}
\label{sec:results-discussion}

We noticed many branch-and-bound nodes being generated,
suggesting that a smarter and more effective bounding technique to
generate an effective initial dual bound could speed up the convergence
of the MIP solver by reducing the number of branching occurrences.

Heuristic algorithm to solve maxflow???

\chapter{Results}
\label{sec:results}

\section{CVRP Benchmark Instances}
\label{sec:results-benchmark-instances}

\mytodo{Introduce the original benchmark used and their naming convention, and filename convention <F>-n<X>-k<Y>}

\mytodo{Talk about how benchmark instances are obtained}

\mytodo{Say that they follow the TSPLIB95 file format}

\mytodo{Say that the distances are rounded. Refer to \cite{uchoa2017}, which contains a small explanation why this is done}

\begin{comment}
\cite{uchoa2017}
THE CVRPLIB website
The typical instance repository of today is a web page that allows downloading the instance
files and includes additional textual information, like file format description, instance
source, best known/optimal solution values, etc. The CVRLIB web page, where the new
instances (and all the previous CVRP instances described in Section 2) are available
(http://vrp.galgos.inf.puc-rio.br/index.php/en/), is more sophisticated:
\end{comment}

\mytodo{Put all the instances used in tables as is done in \cite{uchoa2017}}

\section{\bapcod}
\label{sec:results-bapcod}

\textit{\bapcod}\ \parencite{sadykov2021} is a software package,
developed in France at the Bordeaux University and Bordeaux Research Center ,
which implements a branch-cut-and-price algorithm
and a column generation procedure.
\bapcod takes as input a mixed integer programming model, and solves it by applying
a Dantzig-Wolfe reformulation \parencite{dantzig1960}.
The branch-and-price framework is generic and may be used standalone.
In fact, it supports user developed extensions,
which can provide custom cutting-planes, branching-decisions
or user-defined pricers.
The \textit{VRPSolver} extension \parencite{pessoa2020a}, is
a \bapcod\ extension distributed by the same authors.
This extension implements an RCSP pricer, some cutting-planes
and branching decisions to efficiently solve routing-like problems,
such as the CVRP, VRPTW, but also other problems that exhibit similar
structures (see \cite{pessoa2020a}).
For more detail about \bapcod and its VRPSolver extension refer to
\textcite{sadykov2021}.

\bapcod\ along with its VRPsolver are currently the leading state-of-the-art
technology for solving vehicle routing problems \parencite{pessoa2020a}.

\section{Evaluation Method}
\label{sec:results-evaluation-method}

\section{Performance profiles}
\label{sec:results-performance-profiles}

\mytodo{Explain what are performance profiles and how to interpret them}

\section{Empirical Results}
\label{sec:results-empirical-results}
\mytodo{Spam this section with performance profiles}

\section{Discussion}
\label{sec:results-discussion}

We noticed many branch-and-bound nodes being generated,
suggesting that a smarter and more effective bounding technique to
generate an effective initial dual bound could speed up the convergence
of the MIP solver by reducing the number of branching occurrences.

Heuristic algorithm to solve maxflow???

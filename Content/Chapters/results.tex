\chapter{Results}
\label{sec:results}

\section{Benchmark Instances}
\label{sec:results-benchmark-instances}


\mytodo{Introduce the original benchmark used and their naming convention}

\mytodo{Talk about how benchmark instances are obtained}

\mytodo{Say that they follow the TSPLIB95 file format}

\mytodo{Say that the distances are rounded. Refer to \cite{uchoa2017}, which contains a small explanation why this is done}


\begin{comment}
\cite{uchoa2017}
THE CVRPLIB website
The typical instance repository of today is a web page that allows downloading the instance
files and includes additional textual information, like file format description, instance
source, best known/optimal solution values, etc. The CVRLIB web page, where the new
instances (and all the previous CVRP instances described in Section 2) are available
(http://vrp.galgos.inf.puc-rio.br/index.php/en/), is more sophisticated:
\end{comment}

\mytodo{Put all the instances used in tables as is done in \cite{uchoa2017}}


\section{Performance profiles}
\label{sec:results-performance-profiles}

\mytodo{Explain what are performance profiles and how to interpret them}

\section{Empirical Results}
\label{sec:results-empirical-results}

\mytodo{Spam this section with performance profiles}

\section{Discussion}
\label{sec:results-discussion}

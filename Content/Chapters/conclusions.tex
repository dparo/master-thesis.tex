\chapter{Conclusions}
\label{sec:conclusions}

We proposed a branch-and-cut framework for solving the pricing problem
induced by Capacitated Vehicle Routing Problems (CVRPs)
using a Column Generation (CG) approach.

The first portion of the thesis provided a theoretical foundation for the CVRP
while examining notable contributions to CVRP's exact algorithms.
The dynamic-programming-based label-correcting algorithm,
proposed in \textcite{desrochers1992, feillet2004},
is yet to this day a fundamental component for solving the pricing problem
in contemporary VRP solvers.
We examined the label-correcting algorithm
and discussed additional contributions in the pricing problem domain.

\medskip

We identified the labeling algorithm's limitations in solving pricing problems
with non-stringent vehicle capacities
and its inability to scale to multiple machine cores.
As a solution, we proposed a BAC algorithm to address the pricing problem,
the implementation of which was provided in \cref{sec:implementation-chapter}.
The BAC algorithm was built on top of the CPLEX MIP optimizer,
with the added benefit of scaling to multiple machine cores effortlessly.

Despite the inherent operational differences between the two approaches, we conducted an empirical study in \cref{sec:results}
to assess their performance as the associated CVRP vehicle capacity increases.
Our analysis revises and greatly supplements the previous study published in \textcite{jepsen2014}.

\medskip

The empirical results were discussed in \cref{sec:results-empirical-results,sec:results-discussion}.
While we did not achieve outstanding results in all cases,
we demonstrated that the BAC framework outperforms
the labeling algorithm as the vehicle capacity bound increases.
Our results imply that further research on BAC-based pricer approaches
may provide additional benefits to modern VRP solvers.
BAC-based pricers may be integrated within branch-and-price (BAP) frameworks
supplementing the traditional dynamic programming labeling algorithm.

\section{Improvements and Future Work}
\label{sec:conclusions-improvements-and-future-work}

In the section \cref{sec:conclusions-improvements-and-future-work},
we discuss improvements and future work regarding our BAC-based pricer.

We start by discussing some possible improvements.

The labeling algorithm for pricing $q$-routes of \textcite{desrochers1992} could be used
to provide a better warm starting approach to the CPLEX MIP optimizer.

Mathheuristics \parencite{fischetti2018},
such as local branching \parencite{fischetti2003} or hard fixing,
could be employed to transform an exact BAC algorithm to a heuristic one.
Mathheuristics reduce the size a problem
by starting from a feasible solution and imposing additional invalid constraints.
These constraints typically limit the search-space to solutions
solutions resting "closely" to the current incumbent.
Mathheuristics could also be employed to drive a faster convergence
toward sensible dual bounds in reduced computation time.
Having good dual bounds reasonably quickly reduces the number of branch-and-bounds produced,
which improves memory consumption.
In the best-case scenario, it may also reduce the overall running time.
The utility of mathheuristics at pricing should be investigated further.

Possible improvements:
\begin{itemize}
	\item Measure the running time required to close the root node of the BPC framework.
	      This requires developing the pricer as \bapcod{} plugin.
	\item Algorithmic enhancements.
	\item Improvements on implementation running time and memory efficiency.
\end{itemize}

\medskip

In the future, it would be interesting to investigate the competitiveness of our approach
by instead measuring the time required to solve the entire CVRP instance
with the BPC algorithm using the BAC algorithm as a pricer.
Doing so would require additional implementation efforts to port the BAC-pricer to \bapcod{},
requiring the development of a C++ translation layer
to route the \bapcod{} pricing requests to our BAC pricer.

As a final side-note, it would be interesting to see if
the same results obtained for the CVRP also apply to the VRPTW.
However, this scenario would necessitate radical modifications
to the pricer implementation, the MIP model, and the cutting planes.
In the VRPTW, the MIP model needs to be modified to account for
time window slots.
Time window slots must be modeled as big-M constraints,
which are known to be computationally unstable \parencite{jepsen2008branchandcut}.
Despite the VRPTW is beyond the scope of this thesis,
it could be an interesting subject for studying BAC-based pricers in the future.

%%%%%%%%%%%%%%%%%%%%%%%%%%%%%%%%%%%%%%%%%%%%%%%%%%%%%%%%%%%%%%
%%%%%%%%%%%%%%%%%%%%%%%%%%%%%%%%%%%%%%%%%%%%%%%%%%%%%%%%%%%%%%
%%%%%%%%%%%%%%%%%%%%%%%%%%%%%%%%%%%%%%%%%%%%%%%%%%%%%%%%%%%%%%
%%%%%%%%%%%%%%%%%%%%%%%%%%%%%%%%%%%%%%%%%%%%%%%%%%%%%%%%%%%%%%

% Old crap

%%%%%%%%%%%%%%%%%%%%%%%%%%%%%%%%%%%%%%%%%%%%%%%%%%%%%%%%%%%%%%
%%%%%%%%%%%%%%%%%%%%%%%%%%%%%%%%%%%%%%%%%%%%%%%%%%%%%%%%%%%%%%
%%%%%%%%%%%%%%%%%%%%%%%%%%%%%%%%%%%%%%%%%%%%%%%%%%%%%%%%%%%%%%
%%%%%%%%%%%%%%%%%%%%%%%%%%%%%%%%%%%%%%%%%%%%%%%%%%%%%%%%%%%%%%

The branch-and-cut frameworks can also be easily extended
to become a heuristic pricer
(see mathheuristics: adding an invalid constraint to speed up the convergence of the MIP solver)
by employing
solutions such as hard-fixing.
Hard fixing is a form of randomized diving,
where given any feasible solution,
we fix randomly some variables and we repeat the resolution
process.
Soft-fixing, base on local branching). is another method
to extend a MIP formulation and solver to provide
heuristic solutions.
In local-branching a infeasible extra-constraints is added
which imposes that a new optimal solution
is not far a part from a provided feasible solution.
This heavily prunes the search space, but still
allows the MIP solver to decide which variables
to fix as to provide the best outcome possible.

Implement smarter branching schemes, such as branching over cut-sets.
Study more effective cutting planes and
faster separation techniques.
Study more effective bounding procedures and cutting off values
prior to starting the resolution process through the MIP.

Use min-cut heuristic procedures to improve
the running time of the fractional separation
such as is done \textcite{kernighan1970}.
Such heuristic procedure may overestimate the maxflow
value thus the fractional separation procedure may
become blind to certain violated inequalities that were
there, but couldn't be computed due to the heuristic procedure.
Since the heuristic is faster than the exact push-relabel
algorithm ($O(N^{2.2})$), it may be run on each fractional
separation.

Improve performance running time by employing
sparser GSEC formulation of \labelcref{eq:cptp-gsec-constraints-v2}
whenever possible.

Other improvements: improve the initial dual bound of the branch-and-cut
algorithm by pricing an initial q-route using the labeling algorithm,
or by using the branch-and-cut itself.

Another possible solution could be to emulate the decremental state-space
relaxation, but by employing the branch-and-cut frameworks itself.
The branch-and-cut algorithm is asked to first generate $q$-routes by
relaxing the bounds on the $y_i$ MIP variables,
and then for each vertex visited twice of more, the associated bound
on the $y_i$ is strengthened.

Try out different extended formulations (polynomial number of constraints)
by readjusting the formulations originally developed for the TSP:
the sequential style formulations (MTZ) presented in \textcite{miller1960},
the single commodity flow style (FLOW-1) presented i n \textcite{gavish1978travelling},
or the multi-commodity flow style formulation (MCF) presented in \textcite{wong1980integer,claus1984new}.
Despite trying different formulations may seem a good candidate for improving
the performance of the branch-and-cut pricer,
it has to be noted that in the work of \textcite{taccari2016},
these extended formulations have been shown empirically
to be unsatisfactory when applied to the ESPP (without resource constraints).
Therefore, an hybrid approach between an exteded formulation
and coupled with an effective GSEC separation procedure may improve
the effectiveness of the branch-and-cut pricer.

Create effective preprocessing algorithms to reduce the search space
that need to be explored from the branch-and-cut algorithm.
Develop an effective (meta-)heuristic algorithm on the dual problem,
to feed the branch-and-cut algorithm with an initial good dual bound.

Improve the running time of the GSEC separation by employing an heuristic
Dijkstra algorithm in the support graph induced from the non-zero fractional solutions $x^\star_e > 0 \quad \forall e \in E$.
The min-cut and Gomory-Hu tree algorithms may be employed only if the first approach
doesn't find any violated inequality.

Develop novel and effective cutting planes for the CPTP problem by taking
insipiration from similar problem domains.

Modify the formulations and the valid inequalities as to generate non-elementary routes.
Through a similar approach used in DSSR \parencite{boland2006, righini2008, martinelli2014},
introduce partially cutting planes through integer and fractional separations.
This should allow to close the duality gap quicker.

Employ a heuristic procedure based on
the application of a truncated labeling algorithm
to seek for non-elementary SPPRC, which can be used as valid initial
dual bounds for the branch-and-cut approach.

\medskip

We noticed many branch-and-bound nodes being generated,
suggesting that a smarter and more effective bounding technique to
generate an effective initial dual bound could speed up the convergence
of the MIP solver by reducing the number of branching occurrences.

Heuristic algorithm to solve maxflow???

Possibly improve converge speed by using heuristic patching at each
node of the branch-and-bound tree.

Furthers study, develop new (strong) cutting planes approaches,
that achieve a good trade off between their separation and gain
in the dual-bound improvement.

Extend the branch-and-cut algorithm
to the general ESPPRC.
Problems in IP formulation:
requires direct network case (thus less compact model, unfortunate since compact formulation are
preferred for branch-and-cut algorithms),
computational stability issues in handling big-M
for handling resource constraints at the edge or node level (such as time windows constraints),
see \textcite{jepsen2008branchandcut}.

Note that the separation of additional valid inequalities do not
necessarily need an exact separation procedure.
For these reasons, heuristic separation procedures for the GLM and RCC
cuts may provide a faster convergence of the solution approach.

Branch-and-cut algorithms issue, for some instances (especially on the F-n135-k7 instance)
memory exhaustion of the host machine is a big concern in using such approach.

\begin{comment}
\cite{jepsen2014}
Furthermore, a comparison with state-of-the-art dynamic programming algorithms has
shown that the BAC algorithm is competitive, and acts as a good complement to the dynamic programming algorithms.
That is, in some case the dynamic programming algorithms are much faster and able to solve instances that cannot be
solved by the BAC algorithm. On the other hand the BAC algorithm appear to be superior on very large instances, e.g., the
BAC algorithm solved instances with up to 800 nodes compared to a maximum of 200 nodes for the dynamic programming
algorithms. There is a tendency towards, that the BAC algorithm is faster than the dynamic programming algorithms on
instances with highly negative weights and is slower on instances with solution values close to 0. This is not too surprising
since the bounding functions used in the dynamic programming algorithms is expected to cut off large parts of the state-
space in the latter case. This may render the BAC algorithms less efficient in a column generation scheme for current state-
of-the-art algorithms, since instances with much negativity are usually solved heuristically and only the instances with cost
near zero are solved to optimality. However, our experiments indicate that the BAC algorithm may prove to be worthwhile
when the number of nodes increases (to more than the current maximum of 151 nodes for the CVRP). To sum up, the BAC
algorithm solved 58 out of a total 76 instances which is 18 more than the dynamic programming algorithms, and the BAC
algorithm appears to have its strength when the number of nodes is large.
\end{comment}

\begin{comment}
\cite{jepsen2014}
We believe that there is a large unexploited potential in BAC algorithms for CPTP, and hence the paper is intended to
serve as a platform for further development, also acting as a survey/tutorial. Future research can experiment with subsets
of the presented cuts, extensions of the cuts, and the interplay between these.
\end{comment}

\begin{comment}
\cite{jepsen2011}
Although the main focus of this thesis is exact solution methods, it is impor-
tant to remember that in real life the computational time needed to find the
optimal solution is not always available therefore it is important that the
exact solution algorithms can find good solutions within reasonable time.
Danna and Le Pape [18] have shown how to integrate the Branch-and-Price
algorithm with a local search framework. This integration helps the bcp
algorithm with finding good integer solutions in the early stages of the al-
gorithm. The method has show to result in reasonable good solutions for
the vrptw. Prescott-Gagnon et al. [55] have improved the heuristic ap-
proach for bcp algorithms further by integrating the bcp algorithm with
large neighbourhood search. For bac algorithms methods such as local
branching introduced by Fischetti and Lodi [28] and the feasibility pump
introduced by Fischetti et al. [29] can be used to find fast and good solu-
tions. The main benefit of the exact solution approach is that it provides
both an upper and lower bound.
Though when a fast good solution is needed a heuristics such as the
adaptive large scale neighbourhood search by Pisinger and Ropke [54] for
cvrp, vrptw and many other Vehicle Routing variants or the local search
heuristic by Zachariadis and Kiranoudis [65] are preferable.
\end{comment}

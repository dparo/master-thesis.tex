\chapter{Conclusions and Future work}
\label{sec:conclusions-and-future-work}

The advantage of branch-and-cut software packages is that
they allow for heavy use of parallelization basically for free,
while current label-setting algorithms are single-threaded.
The branch-and-cut frameworks can also be easily extended
to become a heuristic pricer by employing
solutions such as hard-fixing.
Hard fixing is a form of randomized diving,
where given any feasible solution,
we fix randomly some variables and we repeat the resolution
process.
Soft-fixing, base on local branching). is another method
to extend a MIP formulation and solver to provide
heuristic solutions.
In local-branching a infeasible extra-constraints is added
which imposes that a new optimal solution
is not far a part from a provided feasible solution.
This heavily prunes the search space, but still
allows the MIP solver to decide which variables
to fix as to provide the best outcome possible.

Implement smarter branching schemes, such as branching over cutsets.
Study more effective cutting planes and
faster separation techniques.
Study more effecitve bounding procedures and cutting off values
prior to starting the resolution process through the MIP.


Use min-cut heuristic procedures to improve
the running time of the fractional separation
such as is done \textcite{kernighan1970}.
Such heuristic procedure may overestimate the max-flow
value thus the fractional separation procedure may
become blind to certain violated inequalities that were
there, but couldn't be computed due to the heuristic procedure.
Since the heuristic is faster than the exact push-relabel
algorithm ($O(n^2.2)$), it may be run on each fractional
separation.

Other improvements: improve the initial dual bound of the branch-and-cut
algorithm by pricing an initial q-route using the labeling algorithm,
or by using the branch-and-cut itself.

Another possible solution could be to emulate the decremental state space
relaxation, but by employing the branch-and-cut frameworks itself.
The branch-and-cut algorithm is asked to first generate $q$-routes by
relaxing the bounds on the $y_i$ mip variables,
and then for each vertex visited twice of more, the associated bound
on the $y_i$ is strengthened.


\subsection{Future work}
\mytodo{Extend the CPTP model to consider VRPTW, and re-evaluate the empirical results}

\begin{comment}
\cite{jepsen2011}
Although the main focus of this thesis is exact solution methods, it is impor-
tant to remember that in real life the computational time needed to find the
optimal solution is not always available therefore it is important that the
exact solution algorithms can find good solutions within reasonable time.
Danna and Le Pape [18] have shown how to integrate the Branch-and-Price
algorithm with a local search framework. This integration helps the bcp
algorithm with finding good integer solutions in the early stages of the al-
gorithm. The method has show to result in reasonable good solutions for
the vrptw. Prescott-Gagnon et al. [55] have improved the heuristic ap-
proach for bcp algorithms further by integrating the bcp algorithm with
large neighbourhood search. For bac algorithms methods such as local
branching introduced by Fischetti and Lodi [28] and the feasibility pump
introduced by Fischetti et al. [29] can be used to find fast and good solu-
tions. The main benefit of the exact solution approach is that it provides
both an upper and lower bound.
Though when a fast good solution is needed a heuristics such as the
adaptive large scale neighbourhood search by Pisinger and Ropke [54] for
cvrp, vrptw and many other Vehicle Routing variants or the local search
heuristic by Zachariadis and Kiranoudis [65] are preferable.
\end{comment}

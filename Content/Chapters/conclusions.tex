\chapter{Conclusions}
\label{sec:conclusions}

We proposed a branch-and-cut framework for solving the pricing problem
induced by Column Generation (CG) approaches when applied to
Capacitated Vehicle Routing Problems (CVRPs).

The first portion of the thesis provided a theoretical foundation for the CVRP
while examining notable contributions to CVRP's exact algorithms.
The dynamic-programming-based label-correcting algorithm,
proposed in \textcite{desrochers1992, feillet2004},
is yet to this day a fundamental component for solving the pricing problem
in contemporary VRP solvers.
We examined the label-correcting algorithm
and discussed additional contributions in the pricing problem domain.

\medskip

We identified the labeling algorithm's limitations in solving pricing problems
with non-stringent vehicle capacities
and its inability to scale to multiple machine cores.
As a solution, we proposed a BAC algorithm to address the pricing problem,
the implementation of which was provided in \cref{sec:implementation-chapter}.
The BAC algorithm was built on top of the CPLEX MIP optimizer,
with the added benefit of scaling to multiple machine cores effortlessly.

Despite the inherent operational differences between the two approaches, we conducted an empirical study in \cref{sec:results}
to assess their performance as the associated CVRP vehicle capacity increases.
Our analysis revises and vastly supplements the previous study published in \textcite{jepsen2014}.

\medskip

The empirical results were discussed in \cref{sec:results-empirical-results,sec:results-discussion}.
While we did not achieve outstanding results in all cases,
we demonstrated that the BAC framework outperforms
the labeling algorithm as the vehicle capacity bound increases.
Our results imply that further research on BAC-based pricer approaches
may provide additional benefits to modern VRP solvers.
BAC-based pricers may be integrated within branch-and-price (BAP) frameworks
supplementing the traditional dynamic programming labeling algorithm.

\section{Improvements and Future Work}
\label{sec:conclusions-improvements-and-future-work}

In the section \cref{sec:conclusions-improvements-and-future-work}, we discuss improvements and future work regarding our BAC-based pricer.

The labeling algorithm for pricing $q$-routes of \textcite{desrochers1992} could be integrated to provide sensible dual bounds to the CPLEX MIP optimizer in the early stages.
Preprocessing algorithms may also be studied and implemented to reduce the search-space size of the CPTP.

Mathheuristics \parencite{fischetti2018}, such as \textit{local branching} \parencite{fischetti2003} or \textit{hard-fixing}, could be used to transform an exact BAC algorithm into a heuristic one.
Mathheuristics shrinks the search space by imposing additional invalid constraints.
These constraints typically limit the search space to solutions resting "closely" to a candidate point.
The utility of mathheuristic approaches should be investigated further in the context of pricing.

Readjusting the \textit{decremental state-space relaxation} (DSSR) of \textcite{boland2006, righini2008, martinelli2014} to a BAC approach, may also be a viable option for accelerating convergence towards optimal solutions.
Initially, the BAC-pricer could be asked to generate non-elementary $q$-routes.
The elementarity condition could then be gradually enforced over time using an integral cutting-planes approach by strengthening the bounds associated with the $y_i \quad \forall i \in V$ decision variables that violated the elementarity condition.

Porting the ng-routes relaxation \parencite{baldacci2011} to a BAC framework could be an intriguing research topic, though it may be unrealistic to accomplish.


It will be interesting to see if paradigm shifts from the employed CPTP MIP model could provide performance boosts to the BAC pricer. Compact formulations (polynomial number of constraints) could be studied in the context of pricing: such as sequential formulation (MTZ) \parencite{miller1960}, single-commodity flow formulations (FLOW-1) \parencite{gavish1978travelling} or multi-commodity flow formulations (MCF) \parencite{wong1980integer,claus1984new}.
Compact formulations, on the other hand, were empirically proven to be unsatisfactory in \textcite{taccari2016} for the ESPP. Nonetheless, hybridizations of these models, combining decision variables and cutting planes strategies, could be a topic of future evaluation.


%%%%
%%%% Continue review from here
%%%%


The BAC pricer could benefit from novel cutting planes with shorter separation times that achieve reasonable dual-bound improvements.
We did not use any custom branching strategies in our implementation.
Other branching schemes, such as branching over cut-sets of Rounded Capacity Constraints (RCC), could be advantageous to test.
To reduce memory consumption and improve efficiency, we could also tune the cutting planes violation thresholds and use the GSEC sparser formulation of \labelcref{eq:cptp-gsec-constraints-v2} whenever possible.


We saw the impact of fractional labeling and fractional separation in \cref{sec:results-empirical-results,sec:results-discussion} and how these two components are critical to the BAC pricer's efficiency.
The speed of fractional labeling could be increased by using heuristics to compute reasonable min-cuts.
The \textit{Lin-Kernighan heuristic} \parencite{kernighan1970}, for example, could be used to compute reasonable min-cuts in $O(N^{2.2})$ time.
Instead, a simpler heuristic alternative is to perform a DFS traversal in the support graph generated by the non-zero fractional solutions $x^\star_e > 0 \quad \forall e \in E$.
If the heuristic fractional labeling cannot identify any violated inequality, then the exact fractional labeling of \cref{sec:impl-labeling-fractional-solutions} can be used instead.


\medskip

In the future, it would be interesting to investigate the competitiveness of our approach by instead measuring the time required to solve the entire CVRP instance with the BPC algorithm using the BAC algorithm as a pricer.
Doing so would require additional implementation efforts to port the BAC-pricer to \bapcod{}, requiring the development of a C++ translation layer to route the \bapcod{} pricing requests to our BAC pricer.

As a final side-note, it would be interesting to see if the same results obtained for the CVRP also apply to the VRPTW.
However, this scenario would necessitate radical modifications to the pricer implementation, MIP model and cutting planes.
In the VRPTW, the MIP model needs to be modified into an ESPPRC to account for time window slots.
Time window slots must be modeled as big-M constraints, which are known to be computationally unstable \parencite{jepsen2008branchandcut}.
Despite the VRPTW is beyond the scope of this thesis, it could be an interesting subject for studying BAC-based pricers in the future.

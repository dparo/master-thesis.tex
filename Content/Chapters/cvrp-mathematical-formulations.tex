\chapter{Mathematical formulations for the CVRP}
\label{sec:cvrp-mathematical-formulations}

Integer Programming (IP) is a mathematical optimization tool
that can describe combinatorial optimization problems
using constraints, typically defined through linear inequalities.
The constraints limit the optimization problem's solution space.
IP-based formulations also include a list of decision variables
and a linear objective function to be optimized.

\medskip

While we have already described the CVRP verbally in \cref{sec:introduction-chapter},
the CVRP is traditionally defined more rigorously through
(Mixed) Integer Programming (MIP, IP) formulations,
sometimes also known simply as models.
We present three commonly used CVRP mathematical models:
\begin{enumerate}
	\setlength{\itemsep}{0pt}
	\setlength{\parskip}{0pt}

	\item The two-index arc flow (2F) formulation \parencite{laporte1983, laporte1985, laporte1986}
	      is presented in \cref{sec:cvrp-two-index-flow-formulation}.
	\item The three-index arc flow (3F) formulation \parencite{golden1977}
	      is presented in \cref{sec:cvrp-three-index-flow-formulation}.
	\item The Set Partitioning (SP) formulation \parencite{balinski1964}
	      is presented in \cref{sec:set-partitioning-formulation}.
\end{enumerate}

The first two formulations, traditionally designed for branch-and-cut (BAC) algorithms, employ a polynomial number of variables but use an exponential number of constraints.
Instead, the latter formulation employs an exponential number of decision variables.
Despite its age, the SP formulation has become the de facto state-of-the-art core ingredient for solving CVRP problems in recent years, thanks to algorithmic advances and the development of efficient branch-and-price (BAP) frameworks (see \cite{pessoa2020}).
\Textcite{baldacci2004} contains another less ordinary formulation based on a two-commodity flow description.

While the first two formulations will not be the primary focus of this thesis,
they are necessary for understanding some crucial details.
We will predominantly focus our efforts
on the SP formulation. See \cref{sec:branch-and-price}.

We begin by defining some basic notation and mathematical quantities in
\cref{sec:cvrp-mathematical-notation} that will be used throughout the rest of the thesis.

\section{Mathematical notation}
\label{sec:cvrp-mathematical-notation}

The CVRP is traditionally described as a node routing problem modeled through a symmetric and complete graph, where: (i) the vertices of the network represent the customers and the depot, (ii) the edges represent road interconnections.
Let $G = \Tuple*{V, E}$ denote a \textbf{complete undirected network} where:
$V = \Set*{0, 1, \dots, N - 1}$ denotes the set of nodes,
$E = \Set*{e = \Set*{i, j} = \Set*{j, i} \mid \allowbreak i,j \in V, \allowbreak i \ne j}$ the set of undirected edges,
and $N$ the total number of nodes in the network.
The value $0 \in V$ is used to denote the depot node.
The edge set $E$ has size $|E|$, which can be computed through combinatorial enumeration: $|E| = \frac{N (N+1)}{2}$.
For convenience, we define $V_0 = V \setminus \Set*{0}$ to express the set of customers,
and $N_0 = N - 1$ to denote the total number of customers.
Let $\delta(S)$ with $S \subseteq V$ denote
the edges crossing the set $S$ and its complement $\overline{S} = V \setminus S$.
More formally we can express $\delta(S)$ as
$\delta(S) = \Set*{ e = \Set*{i, j} = \Set*{j, i} \in E \mid i \in \Expr*{S \cap V}, j \in \Expr*{ \overline{S} \cap V } }$.
For brevity, we also define $\delta(i) = \delta(\Set*{i})$
to denote the set of edges incident to node $i \in V$.
We also define $E(S) = \Set*{e = \Set*{i, j} = \Set*{j, i}\in E \mid i, j \in S}$
to denote the set of edges having both end points in set $S \subseteq V$.

Let $q_{i} \in \R,\ q_{i} \ge 0 \quad \forall i \in V$ denote the demand function, which represent the required demand that need to be served for vertex $i \in V$.
For convenience, we use a fictitious demand for the depot: $q_0 = 0$.
Given a set $S \subseteq V$, we define $q(S) = \sum_{i \in S} q_i$ as the total demand of the set $S \subseteq V$.
Let $c_{ij} \in \R, c_{ij} > 0$ denote the distance function between a pair of nodes  $i, j \in V$.
The loop arcs $\Set*{i, i} \notin E$ in CVRP are traditionally not allowed, thus we fix $c_{ii} = \infty$.
We assume a Euclidean CVRP problem, i.e. the distance function is symmetric $c_e = c_{ij} = c_{ji}$
and satisfies the triangle inequality $c_{ij} \le c_{ih} + c_{hj}$.
We are also given the total number of identical trucks $K \in \N_+$
and an upper bound $Q \in \R_+$ representing the capacity of each truck.
For convenience of notation, we also define $\mcK = \Set*{1, \dots, K}$ to denote the set of trucks.
Given a set $S \subseteq V_0$,
we denote by $r(S)$ as the minimum number of vehicles required to serve all customers $i \in S$.
The value of $r(S)$ can be obtained by solving
an NP-hard \textit{Bin Packing Problem} (BPP) associated with the CVRP and set $S$.
As we will see later,
it is often simpler to link $r(S)$ to the trivial lower bound of the BPP \parencite{martello1990, martello1990knapsack}:
\begin{equation}
	r(S) \ge \ceil*{\frac{q(S)}{Q}}.
\end{equation}

A route $p$ is a loop-back sequence $p = \Tuple*{p_0, p_1, \dots, p_u, p_{u + 1}}$,
with $p_0 = p_{u + 1} = 0$
in which $\Set*{p_1, \dots, p_u} \subseteq V_0$ customers are visited.
The route $p$ has a travel cost of $c_p = c(p) = \sum_{i=0}^{u} c_{p_i,p_{i+1}}$
with resource consumption $q_p = q(p) = \sum_{i=0}^{u} q_{i}$.
A feasible solution to a CVRP problem consists of \textbf{exactly} $K$ routes (or circuits) $p_k$
associated with each vehicle $k \in \mcK$ starting and ending at the depot node,
where:
(i) each customer is visited once (elementarity condition)
and (ii) the demand covered by each route $p_k$ does not exceed the vehicle capacity $q(p_k) \le Q \quad \forall k \in \mcK$.
An optimal solution to the CVRP minimizes the sum of the overall edge weights across all tours.
The traditional definition of CVRP requires that all vehicles be fully utilized.
In other words, each $p_k$ route must serve at least one customer.
However, we note that using fewer vehicles may sometimes yield better solutions.

\medskip

The TSP can be considered a special case of CVRP where $Q \ge q(V)$ and $K = 1$.
Therefore, all the relaxations and many results obtained for the TSP
are valid, or at least extendable, to the CVRP.

\medskip

Some variants of the basic version of the CVRP,
not considered in this thesis,
allow using only a subset of the total available vehicles,
or consider a heterogeneous fleet characterized by different capacities $Q_1, \dots, Q_k$.
The remainder of this section
introduces the two most common IP mathematical formulations for the CVRP's classical version.

\section{The two-index arc flow (2F) formulation}
\label{sec:cvrp-two-index-flow-formulation}

The two-index arc flow (2F) formulation
was  first presented in \textcite{laporte1983, laporte1985} for the symmetric case,
and later generalized to the undirected case in \textcite{laporte1986}.

We define a set of integer decision variables $x_e \in \Set*{0, 1, 2}$ to indicate the number of times
a vehicle traverses each edge $e \in E$ in the optimal solution.
The model contains $O(N^2)$ integer variables.
The two-index arc flow (2F) formulation can be described as an Integer Program (IP):
\begin{align}
	\min_{x} \quad z_\mt{2F}(x) & = \sum_{e \in E} c_e x_e                 & \label{eq:two-index-flow-obj-func}                                                   \\
	                            & \sum_{e \in \delta(i)} x_e = 2           & \quad \forall i \in V_0 \label{eq:two-index-flow-two-edges-incident-per-customer}    \\
	                            & \sum_{e \in \delta(0)} x_e = 2 K         & \label{eq:two-index-flow-two-k-edges-incident-in-the-depot-node}                     \\
	                            & \sum_{e \in \delta(S)} x_e \ge 2 r(S)    & \quad \forall S \subseteq V_0,\ |S| \ge 1 \label{eq:two-index-flow-ccc}              \\
	                            & x_e                   \in \Set*{0, 1, 2} & \quad \forall e \in \delta(0) \label{eq:two-index-flow-x-mip-var-bounds-depot}       \\
	                            & x_e                   \in \Set*{0, 1}    & \quad \forall e \in E \setminus \delta(0) \label{eq:two-index-flow-x-mip-var-bounds}
\end{align}
where $z_\mt{2F}(x)$, as defined in \labelcref{eq:two-index-flow-obj-func}, is the objective function meant to minimize the overall routing cost (travel time).
Constraint \labelcref{eq:two-index-flow-two-edges-incident-per-customer} is the degree constraint which imposes flow conservation: exactly two incident edges must be picked for each customer.
Constraint \labelcref{eq:two-index-flow-two-k-edges-incident-in-the-depot-node} is the degree constraint at the depot, it forces that exactly $2K$ incident edges at the depot are picked, thus forcing exactly $K$ routes to be constructed.
Constraints \labelcref{eq:two-index-flow-x-mip-var-bounds-depot,eq:two-index-flow-x-mip-var-bounds} forces each edge to be traversed at most once,
except for all edges incident at the depot.
The edge-case modeled in constraint \labelcref{eq:two-index-flow-x-mip-var-bounds-depot} is necessary to allow single-customer routes.
Constraint \labelcref{eq:two-index-flow-ccc}, are the so-called Capacity Cut Constraints (CCC), also called Rounded Capacity Constraints (RCC), they function both:
(i) as Subtour Elimination Constraints (SECs),
by imposing connectivity of the solution by avoiding the formation of spurious unconnected subtours
and (ii) as a capacity constraint,
by imposing that any customer set $S$ is crossed by a number of edges not smaller than $r(S)$.
Remember that $r(S)$ represents the minimum number of vehicles required to serve all customers in a given $S$;
additionally, $r(S)$ always satisfies $r(S) \ge 1$ for non-trivial CVRP instances where $q(S) > 0$.

It was shown in \textcite{martello1990knapsack, cornuejols1993}, that it is possible to replace $r(S)$ in constraint
\labelcref{eq:two-index-flow-ccc}, with the much simpler BPP lower bound $\ceil*{q(S) / Q}$
thus obtaining the following inequality:
\begin{equation}
	\label{eq:two-index-flow-ccc-bpp-lb}
	\sum_{e \in \delta(S)} x_e \ge 2 \ceil*{\frac{q(S)}{Q}}   \quad \forall S \subseteq V_0, |S| \ge 1.
\end{equation}
The looser inequality of \cref{eq:two-index-flow-ccc-bpp-lb} is sufficient
to solve the 2F formulation optimally.
However, a better lower bound for the BPP
can improve the linear relaxation, reducing the number of branching occurrences.

The CCC constraints in \cref{eq:two-index-flow-ccc},
may be transformed in the so called Generalized Subtour Elimination Constraints (GSEC) \parencite{laporte1985},
by means of the degree constraints \labelcref{eq:two-index-flow-two-edges-incident-per-customer,eq:two-index-flow-two-k-edges-incident-in-the-depot-node}:
\begin{equation}\label{eq:cvrp-2flow-gsec}
	\sum_{e \in E(S)} x_{ij} \le |S| - r(S) \quad \forall S \subseteq V_0,\ |S| \ge 1,
\end{equation}
where, again, $r(S)$ may be replaced by the trivial BPP's lower bound $\ceil*{q(S) / Q}$.

The GSECs avoid the formation of spurious unconnected subtours and impose that at least $r(S)$ edges leave set $S$.
The number of GSEC (or CCC) inequalities appear in exponential number in the two-index formulation model,
thus making a direct solution of the linear relaxation impractical.
To overcome this issue, it is possible to avoid adding the GSEC inequalities statically in the model.
Instead, an appropriate cutting-plane algorithm and separation procedure may be employed to generate dynamically
only the necessary (violated) GSEC constraints
during the running time of the branch-and-cut algorithm.

The so-called compact models \parencite{miller1960, christofides1979vehicle, desrochers1991}
make use of a polynomial number of constraints.
Unfortunately, the linear relaxations produced by these compact formulations are significantly weaker.
The SECs are mathematically proven to be "strong" for the TSP:
they are facet-defining for the TSP polytope by uniquely characterizing its convex-hull \parencite{grotschel1975}.
Similar arguments, however, cannot be applied to the CVRP due to its more complicated structure.
Studying the polyhedral properties of even the standard variation of the VRP
has yielded few satisfactory results, see \textcite{campos1991, cornuejols1993}.

Unfortunately, as \textcite{augerat1995} demonstrated,
the separation problem for the CCCs is NP-complete,
limiting the applicability of the 2F formulation for solving the CVRP.
As a result, several authors \parencite{augerat1995, augerat1998, ralphs2003}
developed fractional separation heuristics for the CCC.
However,
an exact CCC separation is still required
for rejecting non-feasible integral solutions
and ensuring the correctness of the approach.
\textcite{lysgaard2003cvrpsep} implements efficient
heuristics for separating the CCC of \cref{eq:two-index-flow-ccc}.

\section{The three-index arc flow (3F) formulation}
\label{sec:cvrp-three-index-flow-formulation}

When it comes to modeling more "colorful" CVRP variants, the two-index arc flow (2F) model lacks sufficient descriptive power.
For example, the simple CVRP variant where the fleet of trucks is heterogeneous and characterized by capacities $Q_1, \dots, Q_K$ cannot be described with the 2F formulation, since there's no definite one-to-one mapping on which truck crosses an edge $e \in E$.

The three-index arc flow (3F) formulation, first presented in \textcite{golden1977}, fixes this issue.
The 3F formulation makes use $O(N^2 K + N K)$ integer decision variables.
A set of integer variables $x_{ek} \in \Set*{0, 1, 2},\ e \in E,\ k \in \mcK$ is used to encode the number of times a truck $k$ traverses an edge $e \in E$.
A new set of binary variables $y_{ik} \in \Set*{0, 1},\ i \in V,\ k \in \mcK$ is used to model whether truck $k$ covers a node $i \in V$.

The three-index arc flow (3F) formulation can be described as an Integer Program (IP):
\begin{align}
	\min_{x, y} \quad z_\mt{3F}(x, y) & = \sum_{k \in \mcK} \sum_{e \in E} c_{e} x_{ek} & \label{eq:three-index-flow-obj-func}                                                                                    \\
	                                  & \sum_{k \in \mcK} y_{ik} = 1                    & \quad \forall i \in V_0 \label{eq:three-index-flow-all-customers-visited}                                               \\
	                                  & \sum_{k \in \mcK} y_{0k} = K                    & \label{eq:three-index-flow-tour-starts-and-ends-at-depot}                                                               \\
	                                  & \sum_{e \in \delta(i)} x_{ek} = 2 y_{ik}        & \quad \forall i \in V,\ \forall k \in \mcK \label{eq:three-index-flow-force-visited-customer-if-flow}                   \\
	                                  & \sum_{i \in V} q_i y_{ik} \le Q                 & \quad \forall k \in \mcK \label{eq:three-index-flow-force-resource-upper-bound}                                         \\
	                                  & \sum_{e \in \delta(S)} x_{ek} \ge 2 y_{hk}      & \quad \forall h \in S,\ \forall S \subseteq V_0,\ |S| \ge 2,\ \forall k \in \mcK \label{eq:three-index-flow-secv1}      \\
	                                  & x_{ek}                   \in \Set*{0, 1, 2}     & \quad \forall e \in \delta(0),\ \forall k \in \mcK             \label{eq:three-index-flow-x-mip-var-bounds-depot}       \\
	                                  & x_{ek}                   \in \Set*{0, 1}        & \quad \forall e \in E \setminus \delta(0),\ \forall k \in \mcK             \label{eq:three-index-flow-x-mip-var-bounds} \\
	                                  & y_{ik}                    \in \Set*{0, 1}       & \quad \forall i \in V,\ \forall k \in \mcK  \label{eq:three-index-flow-y-mip-var-bounds}
\end{align}
where $z_\mt{3F}(x, y)$, as defined in \labelcref{eq:three-index-flow-obj-func}, is the objective function to be minimized (i.e. the overall travel distance).
Constraint \labelcref{eq:three-index-flow-all-customers-visited} forces all customers to be served exactly once.
Constraint \labelcref{eq:three-index-flow-tour-starts-and-ends-at-depot} forces all the truck tours to start at the depot and end at the same spot.
Constraint \labelcref{eq:three-index-flow-force-visited-customer-if-flow} binds the $y_{ik}$ variables to the corresponding $x_{ijk}$ variables, by ensuring that if a truck's route passes through a vertex, then the corresponding node is marked as visited.
Constraint \labelcref{eq:three-index-flow-force-resource-upper-bound} is the capacity upper bound constraint and it ensures that the demand served by each truck does not exceed the truck capacity.
Constraints \labelcref{eq:three-index-flow-secv1} are the Generalized Subtour Elimination Constraints (GSECs), they impose the connectivity of the route and are used to avoid the formation of spurious unconnected subtours.
Constraints \labelcref{eq:three-index-flow-x-mip-var-bounds-depot,eq:three-index-flow-x-mip-var-bounds} forces each edge to be traversed at most once,
except for all edges incident at the depot.
The edge-case modeled in constraint \labelcref{eq:three-index-flow-x-mip-var-bounds-depot}
is necessary to allow single-customer routes.
Finally, constraint \labelcref{eq:three-index-flow-y-mip-var-bounds}
bounds and forces integrality for the $y_{ik}$ binary variables.

Constraint \labelcref{eq:three-index-flow-secv1} may be replaced in an equivalent form
with traditional (non generalized) TSP subtour elimination constraints (see \cite{fisher1981}):
\begin{equation}\label{eq:three-index-flow-secv2}
	\sum_{e \in E(S)} x_{ek} \le |S| - 1 \quad \forall S \subseteq V_0, |S| \ge 2,\ \forall k \in \mcK.
\end{equation}
Since \labelcref{eq:three-index-flow-secv1}, or equivalently \labelcref{eq:three-index-flow-secv2},
are exponential in the number of nodes $N$,
they are usually not inserted statically in the model
but are instead generated lazily within the running time of the resolution process.

The three-index arc flow (3F) model generalizes the two-index (2F) version.
The 2F formulation may be viewed as a special case of the 3F formulation by aggregating
all $x_{ek}$ into a single variable $x_e$:
\begin{equation}
	x_e = \sum_{k \in \mcK} x_{ek} \quad \forall e \in E.
\end{equation}
However, the three-index arc flow formulation suffers a major downside compared to the two-index version.
When modeling homogeneous fleets CVRPs, the 3F model suffers from a multiplicity of the solution space.
In fact, since all the vehicles share the same capacity,
distinct feasible solutions can be obtained through symmetry
by simply permuting the identity $k \in \mcK$ of each truck.

\section{The Set Partitioning (SP) formulation}
\label{sec:set-partitioning-formulation}

The Set Partitioning (SP) formulation,
also known as Path Flow formulation,
is an extended formulation originally presented in \textcite{balinski1964}.
It operates differently from the two/three-index arc flow formulation or many further commonly encountered IP models.
The SP formulation uses a tiny number of constraints
while offloading much of the search-space complexity into an exponential number of binary variables.

The SP formulation can be viewed as a Dantzig-Wolfe decomposition \parencite{dantzig1960}
and commodity aggregation \parencite{desaulniers1998}
of the three-index arc flow formulation.
The Dantzig-Wolfe reformulation approach is an application of a
decomposition principle in which one addresses numerous smaller structured sub-problems
rather than being confronted with the original problem.
This approach is helpful, especially when the
original problem's resolution complexity
exceeds what can be solved in a reasonable amount of time \parencite{vanderbeck2005}.

Let $P = \Set*{p \mid p\ \text{is a single-truck elementary feasible route}}$ be the set of all feasible routes.
Let $\lambda_p \in \Set*{0, 1}$ be a binary variable indicating whether route $p$ is selected.
Let $a_{ip}, a_{ep} \in \Z$ be "static encodings" (integer coefficients)
for a route $p \in P$ that respectively count
the number of times vertex $i \in V$ or an edge $e \in E$ is covered.
Any $p \in P$ is feasible if it satisfies the following two conditions:
$a_{ip} \in \Set*{0, 1} \quad \forall i \in V$ and $q(p) \le Q$.

We recall that $c_p = c(p)$ represents the cost of a feasible route $p \in P$,
which can be trivially computed in $O(N)$.
The SP model forces $K$ routes $\Tuple*{p_1, \dots, p_K} \in P^K$ to be included in the optimal solution.

The SP formulation is described through an Integer Program (IP):
\begin{align}
	\min_{\lambda} \quad z_\mt{SP}(\lambda) & = \sum_{p \in P}  c_p \lambda_p              & \label{eq:set-partitioning-obj-func}                                                       \\
	                                        & \sum_{p \in P} \lambda_{p} = K               & \label{eq:set-partitioning-K-routes}                                                       \\
	                                        & \sum_{p \in P}  a_{ip} \lambda_p = 1         & \quad \forall i \in V_0 \label{eq:set-partitioning-customers-visited-by-exactly-one-route} \\
	                                        & \lambda_p                    \in \Set*{0, 1} & \quad \forall p \in P \label{eq:set-partitioning-lambda-mip-var-bounds}
\end{align}
where, $z_\mt{SP}(\lambda)$, as defined in \labelcref{eq:set-partitioning-obj-func}, is the objective function to be minimized (i.e. the overall travel distance).
Constraint \labelcref{eq:set-partitioning-customers-visited-by-exactly-one-route} forces each customer to be covered by exactly one route.
Constraint \labelcref{eq:set-partitioning-K-routes} enforces that exactly $K$ routes are selected.
Finally, constraint \labelcref{eq:set-partitioning-lambda-mip-var-bounds} is the bounding and integrality constraints for binary variables $\lambda_p \ \forall p \in P$.

The set of feasible routes $P$ may also be extended to include non-elementary routes
without affecting the correctness of the approach.
Thanks to \labelcref{eq:set-partitioning-customers-visited-by-exactly-one-route}
any route $p \in P$ which does not satisfy $a_{ip} \in \Set*{0, 1} \quad \forall i \in V$
will not belong to any optimal solution
regardless of whether the extension is applied or not.

As one might expect, the SP formulation
cannot be instantiated or directly solved
because of the exponential number of binary variables.
A variant of the SP formulation, on the other hand,
can be efficiently addressed by Column Generation (CG) approaches
embedded within a branch-and-price (BAP) framework.
Branch-and-price frameworks and column generation have been applied
with notable success to vehicle routing problems.
Refer to \cref{sec:branch-and-price} for a discussion on
column generation and BAP frameworks applied to the CVRP.

As \textcite{toth2002} point out if the distance matrix satisfies the triangle inequality,
then it is possible to rewrite the SP formulation into an equivalent Set Covering (SC) formulation
by substituting \labelcref{eq:set-partitioning-customers-visited-by-exactly-one-route}
in favor of the simpler inequality:
\begin{equation}\label{eq:set-covering-customers-visited-by-exactly-one-route}
	\sum_{p \in P}  a_{ip} \lambda_p \ge 1  \quad \forall i \in V_0,
\end{equation}
therefore the SC formulation may be expressed as an Integer Program (IP):
\begin{align}
	\min_{\lambda} \quad z_\mt{SC}(\lambda) & = \sum_{p \in P}  c_p \lambda_p              & \label{eq:set-covering-obj-func}                                                       \\
	                                        & \sum_{p \in P} \lambda_{p} = K               & \label{eq:set-covering-K-routes}                                                       \\
	                                        & \sum_{p \in P}  a_{ip} \lambda_p \ge 1       & \quad \forall i \in V_0 \label{eq:set-covering-customers-visited-by-exactly-one-route} \\
	                                        & \lambda_p                    \in \Set*{0, 1} & \quad \forall p \in P \label{eq:set-covering-lambda-mip-var-bounds}.
\end{align}
Under the triangle inequality assumption, any feasible solution for the SC formulation \labelcref{eq:set-covering-obj-func,eq:set-covering-K-routes,eq:set-covering-customers-visited-by-exactly-one-route,eq:set-covering-lambda-mip-var-bounds} is also feasible for the SP formulation \labelcref{eq:mp-obj-func,eq:mp-customers-visited-by-exactly-one-route,eq:mp-K-routes,eq:mp-lambda-mip-var-bounds}.
Transforming the SP to the SC formulation vastly shrinks (halves) the size of the dual solution space.

\medskip

The SP and SC formulation have two main advantages compared to the formulations presented in previous sections.
First, their linear relaxation provides excellent lower bounds \parencite{bramel1997},
compared to the 2F and 3F formulations.
Second, they can handle many VRP variants (and more)
even described through very complex constraints,
since their characterizations are captured within the definition of the set $P$ itself.

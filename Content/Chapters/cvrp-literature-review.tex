\chapter{Literature Review on CVRP exact methods}
\label{sec:intro-literature-review}

This chapter lists some significant contributions regarding exact approaches in the CVRP and VRPTW domains.

\textcite{dantzig1959} were
the pioneers of the contemporary vehicle routing problem two
introduced the CVRP problem to the masses.
A few years later, \textcite{clarke1964} proposes
an effective greedy heuristic algorithm for tackling the CVRP.
The presentation of this new problem led to a whole new branch
of scientific research within the operations research community.
In the last 60 years, a huge number of contributors in
the operations research community has studied and proposed many
mathematical models and algorithms, both exact and (meta-)heuristic,
for solving the VRP.

Until roughly the late 1980s, exact approaches to the CVRP were
based on tree-search algorithms employing branch-and-bound schemes
(see \cite{pierce1969, christofides1969, christofides1981exact, laporte1986}),
occasionally employing Lagrangian duality relaxation (see \cite{fisher1994, miller1995}),
or additive bounding procedures (see \cite{fischetti1994a, hadjiconstantinou1995}).

Some authors applied branch-and-cut (BAC) algorithms  to the VRP from the 1980s to the early 2000s.
\textcite{laporte1983, laporte1985} proposed the two-index arc flow formulation
and a branch-and-cut algorithm for optimally solving the CVRP.
\Textcite{augerat1995approche}, later proposed a branch-and-cut scheme
integrating additional valid inequalities.
A few years later, \textcite{lysgaard2004}
proposed a branch-and-cut framework and
a set of improved separation procedures for the valid CVRP inequalities that were known at the time.
Other notable BAC-based contributions
can be found in \textcite{araqueg1994, augerat1995, achuthan1996, blasum2000, ralphs2003, achuthan2003, baldacci2004}.
\citeauthor{baldacci2004} was able to solve a 135 customers-sized routing problem
from the F test-set  \textcite{fisher1994}
for the first time.

Despite the promising results, nonetheless,
the branch-and-cut contributions of the time demonstrated
that CVRP instances with even a few customers (less than 100)
proved exceedingly challenging to solve exactly.

\Textcite{desrosiers1984, agarwal1989setpartitioningbased}
are a few very early attempts at applying the \textit{Set Partitioning} (SP) formulation
and a \textit{column generation} scheme to routing problems.
The \textit{Set Partitioning} formulation and the \textit{column generation}
will be  presented in more details in \cref{sec:set-partitioning-formulation, sec:column-generation-and-pricing-problem}
respectively.
These authors demonstrated that the column generation scheme proved very satisfactory
for solving the VRPTW under tight constraints
but did not achieve remarkable results for the CVRP at the time.
As a result, until the early 2000s,
branch-and-cut approaches were deemed the best approach for solving the CVRP.

In \citeyear{desrochers1992}, \citeauthor{desrochers1992}
proposed a dynamic programming-based label correcting algorithm for the SPPRC,
and devised it within a \textit{branch-and-price} (BAP) scheme to tackle the VRPTW.
This labeling algorithm is still a key component of modern BPC schemes
for efficiently solving the \texit{pricing problem}.
\textit{Branch-and-price} schemes and the \textit{pricing-problem}
will be both discussed in the dedicated \cref{sec:brancha-and-price}.
The 2-paths inequalities were introduced \textcite{kohl1999},
which were later generalized in \textcite{desaulniers2008}.
\Textcite{kohl1999} was one of the first works to attempt
to integrate cutting planes within a BAP framework for solving the CVRP.

%%%%%%%%%%%%%%%%%%%%%%%%%%%%%%%
%%%%%%%%%%%%%%%%%%%%%%%%%%%%%%% Continue reviewing from here
%%%%%%%%%%%%%%%%%%%%%%%%%%%%%%%

Later, \textcite{baldacci2008}, proposed a slightly different approach w.r.t. \textcite{fukasawa2006}.
They employ a column generation considering elementary routes and additional inequalities.
\citeauthor{baldacci2008} employed a bounding procedure to quickly
improve the dual bound while also reducing the number
of pricing iterations and the number of columns to take into consideration.
Instead of branching, they employed full enumeration of the candidate
routes and use a MIP package to solve the associated problem at the root node.
\textcite{pessoa2008} improves over the work of \textcite{fukasawa2006}
by integrating the work of \textcite{baldacci2008}
through an hybridization between traditional branching and enumeration.
\textcite{contardo2011} instead it takes a different approach.
It considers $q$-routes with 2-cycle eliminations, and non-robust
k-CECs cuts are used to enforce elementarity.

\textcite{baldacci2011} propose the ng-routes, a very effective
partial (soft) elementarity constraint employed in modern VRP solvers,
that trade-offs the elementarity condition (i.e. good dual bound in pricing)
for a simpler pricer problem.
\textcite{ropke2012} revisits the work \citeauthor{fukasawa2006} and applies
the ng-routes relaxation and strong branching schemes.

\textcite{contardo2014} improve over \textcite{contardo2011}
by proposing an effective enumeration scheme and employed
$q$-routes with 2-cycle elimination, ng-routes with ng-set size 8,
non-robust k-CECs cuts to avoid larger cycles
and the decremental state-space relaxation technique.
With this approach they were able to solve CVRP instances made of 151 customers.

\textcite{martinelli2014} extends the decremental state-space relaxation
to handle ng-sets.
In their approach the size of the ng-sets size
is enlarged dynamically in order to seek for more effective dual bounds.

Another modern BCP algorithm was proposed in \textcite{pecin2017improved}.
By using a bidirectional label setting algorithm,
lm-SRCs cuts, modern dual smoothing and enumeration approaches
and the ng-routes relaxation, they were able to solve CVRP instances made of 200 customers
and also some instances up to 360 customers.
\textcite{pecin2017new} extends the work of \textcite{pecin2017improved} to the VRPTW variant.
At the day of writing, \textcite{pessoa2018enhanced, pessoa2020generic} are the current state-of-the-art
BCP frameworks based on column generation with ng-routes relaxation
for solving the CVRP problem.
\textcite{pessoa2020generic} was able to solve to optimality six opened
CVRP instances of up to 548 customers from the instances of family set X \parencite{uchoa2017}.

Thanks to the major advancements on the past decades
on exact BPC algorithms
have led to exact solutions of CVRP problems composed of
more than 300 customers \parencite{costa2019}.

Once we introduce the pricing sub-problem induced by BAP/BPC frameworks
in \cref{sec:branch-and-price},
we will discuss additional contributions regarding this problem domain,
see \cref{sec:solving-the-pricing-problem}.
For a literature review on branch-and-cut approaches for pricing, see later discussion
in \cref{sec:bac-approaches-for-the-pricing-problem}.

\begin{comment}
\cite{jepsen2011}

Before 1980 very few exact algorithms for cvrp and vrptw had been
proposed, but in the early 1980s two new exact methods where proposed.
From this point the history of exact methods for cvrp and vrptw can
be divided into three phases. The first phase was the introduction of the
Set Partition and the development of Branch-and-Cut-and-Price (bp) algo-
rithms using a relaxed pricing problem. The second was the development of
Branch-and-Cut (bac) algorithms. In the current phase the pricing problem
is no longer relaxed and cuts in the master problem of the Branch-and-Cut-
and-Price algorithms is used. The first two phases where started at the same
point in time and there is still development on the algorithms in the context
of cvrp and vrptw. The algorithms from these two phases are also used
on several other variants of the Vehicle Routing Problem. The third phase
was started in the middle of the 2000s and the algorithms from this phase
are currently the best overall performing algorithms.
\end{comment}

\chapter{The Capacitated Profitable Tour Problem (CPTP)}


\begin{comment}
\cite{jepsen2011}
This paper considers the capacitated profitable tour problem (CPTP).
The CPTP belongs to the group of problems known as travelling salesman
problems with profits. In CPTP each customer is associated with a profit
and a demand and the objective is to find a capacitated tour (rooted in
a depot node) that minimizes the total travel distance minus the profit of
the visited customers. The CPTP can be recognized as the sub-problem in
many column generation applications. We present a branch-and-cut algo-
he objective is to find a tour
rooted in the depot where the demand accumulated at the customers does
not exceed the capacity, and the total travel distance subtracting the profits
gained by visiting customers is minimized.

The CPTP is a side-constrained version of the profitable tour problem
named by Dell’Amico et al. [On prize-collecting tours and the asymmetric travelling salesman problem, 1995]
, a problem that falls within the category of
traveling salesman problems with profits as classified by Feillet et al. [Traveling salesman problems with profits, 2005]
Other problems in this category are the orienteering problem (OP) (also
known as the selective travelling salesman problem) and the prize-collecting
travelling salesman problem (PCTSP). In the OP the total tour length is
bounded from above, and the objective is to maximize the profit gained by
visiting customers. In the PCTSP the objective is similar to the CPTP but a
minimum amount of profits must be collected on the tour. In the context of
the capacitated vehicle routing problem (CVRP) the CPTP appears as the
sub-problem in column generation methods, see e.g. Baldacci et al. [3, 2].
In this context, the CPTP is often transformed to a path problem (a path
is obtained from the tour by splitting the depot into two nodes) and is de-
noted the elementary shortest path problem with resource constraints. The
resource is given as an accumulation of demand of the visited customers and
is constrained by the capacity. However, in recent routing applications the
sub-problem is complicated considerably by the introduction of additional
cuts in the column generating master problem, such as the strong capacity
inequalities [2], the subset-row inequalities [20], the Chvátal-Gomory rank-1
cuts [28], and the clique inequalities [33]. The sub-problem can no longer
be considered a CPTP. Moreover, the sub-problems are often solved as fea-
sibility problems instead of optimization problems which may favour other
types of combinatorial algorithms than branch-and-cut algorithms.
Laporte and Martello [21] showed that the OP is NP-hard by reduc-
tion from the Hamilton circuit problem. Using a similar reduction it can
be shown that the CPTP also belongs to the class of NP-hard problems.
If there are no cycles with negative cost in the graph G, then the CPTP
is solvable in pseudo-polynomial time using a dynamic programming algo-
rithm. In this particular case the CPTP relates to the constrained shortest
path problem (again by transformation to a path problem). Several al-
gorithms based on dynamic programming exist for this problem, see e.g.,
Beasley and Christofides [6],Carlyle et al. [9], Dumitrescu and Boland [13],
and Muhandiramge and Boland [25].
Bixby [Polyhedral analysis and effective algorithms for the capacitated vehicle routing problem, 1999]
considers the CPTP in her PhD thesis on the CVRP and present a mathematical model
and a branch-and-cut (BAC) algorithm.  Letch-
ford and Salazar-Gonzalez [24] discuss projection results for the CVRP and
present two families of multistar inequalities that are valid for the CPTP.
Other work on the CPTP in a CVRP context is mainly concerned with
dynamic programming algorithms.
The contribution of this paper is the introduction of an IP model for the
CPTP and a BAC algorithm for solving it. This includes the adaption of
several valid inequalities from e.g. the OP and the CCCP, the introduction
of the rounded multistar inequalities, and a proof of validity for all inequal-
ities with regard to the CPTP. Also, we have successfully implemented a
separation heuristic for finding knapsack large multistar inequalities that
prove their usefulness for the CPTP. The computational experiments show
that the BAC algorithm is competitive with the state-of-the-art dynamic
programming algorithms. In particular, the BAC algorithm is able to solve
instances with 800 nodes to optimality where the dynamic programming
algorithms cannot solve instances with more than 200 nodes. In general
the two algorithms appear to complement each other well. The
\end{comment}


\cite{jepsen2014} provided in their work an IP formulation for the CPTP.
The IP model presented in this section is derived from their work, and it is here summarized.

Let $G = \Tuple*{V, E}$ denote a complete undirected graph, where $V = \Set*{0, 1, \dots, N - 1}$ denotes the set of nodes,
$E = \Set*{e = (i, j) \mid i,j \in V, j \ge i + 1}$ the set of edges, and $N$ the number of nodes in the graph.
The value $0 \in V$ is used to denote the depot node.
For convenience, we define $V_0 = V \setminus \Set*{0}$ to express the set of customers, and $N_0 = N - 1$ to denote the number of customers.
Let $\delta(S)$ with $S \subseteq V$ denote the edges crossing the set $S$ and its complement $\overline{S} = V \setminus S$.
More formally we can express $\delta(S)$ as $\delta(S) = \Set*{ (i, j) \in E \mid i \in \Expr*{S \cap V}, j \in \Expr*{ \overline{S} \cap V } }$.
For brevity, we also define $\delta(i) = \delta(\Set*{i})$ to denote the set of edges incident to node $i \in V$.

Let $p_{i} \in \R, p_{i} \ge 0$ denote the profit function, and $q_{i} \in \R, q_{i} \ge 0$ denote the demand function, which represent respectively the profit gain and required demand in visiting a vertex $i \in V$.
By convention $q_0 = 0$, but notice that the profit at the depot is allowed to be $p_0 \ge 0$.
Let $d_{ij} \in \R, d_{ij} > 0$ denote the distance function between a pair of nodes  $i, j \in V$.
We assume that the distance function is symmetric $d_{ij} = d_{ji}$ and satisfies the triangle inequality $d_{ij} \le d_{ih} + d_{hj}$.

We define two sets of binary variables: $x_{ij}$, $y_{i} \in \Set*{0, 1}$, used to compose output solutions for the CPTP.
Together they model respectively an edge and a vertex as being part of the tour solution.

Given an upper bound $Q \in \R,Q \ge 0$ for the total resource consumption, we can finally express the CPTP problem as an integer programming formulation

\begin{align}
	\min_{x,y} \quad z(x, y) & =  \sum_{i \in V} \sum_{\EqStackTwo{j \in V}{j \ge i + 1}} d_{ij} x_{ij} - \sum_{i \in V} p_i y_i \label{eq:obj-function}                                                                                                                \\
	                         & y_0 = 1                                                                                                                   & \label{eq:depot-part-of-tour-constraint}                                                                     \\
	                         & \ExprCptpEdgesIncident{i}  = 2 y_i                                                                                        & \quad \forall i \in V         \label{eq:flow-conservation-constraint}                                        \\
	                         & \ExprCptpFlowExiting{S} \ge 2 y_{i}                                                                                       & \quad \EqStackTwo{\forall i \in S}{\forall S \subseteq V_0,\ \SetSize*{S} \ge 2} \label{eq:gsec-constraints} \\
	                         & \ExprCptpDemandSum  \le Q                                                                                                 & \label{eq:resource-upper-bound-constraint}                                                                   \\
	                         & x_{ij}                   \in \Set*{0, 1}                                                                                  & \quad \forall (i, j) \in E               \label{eq:x-mip-var-bounds}                                         \\
	                         & y_{i}                    \in \Set*{0, 1}                                                                                  & \quad \forall i \in V,\ i \ge 1          \label{eq:y-mip-var-bounds}
\end{align}

The equation \eqref{eq:obj-function} denotes the objective function, and it is made of two terms: the cost of traveling through the tour (positive contribution) and the total sum of the profits associated to visiting a subset of the vertices (negative contribution).
The constraint \eqref{eq:depot-part-of-tour-constraint} ensures that the depot is always visited.
Constraints \eqref{eq:flow-conservation-constraint} imposes flow conservation constraints for each vertex.
The equation \eqref{eq:gsec-constraints} are the generalized subtour elimination constraints \textbf{(GSECs)} and must be present to guarantee that the solution does not contain spurious unconnected subtours.
There is an exponential number of GSECs, as such, for performance reasons these constraints cannot be inserted statically in a MIP solver, and must instead be efficiently dynamically separated.
The equation \eqref{eq:resource-upper-bound-constraint} models the fact that a tour may visit any vertex as long as the total demand doesn't exceed the vehicle capacity.
Finally, the equations \eqref{eq:x-mip-var-bounds}, \eqref{eq:y-mip-var-bounds} specify the bounds and domain of the $x_{ij}$ and $y_{i}$ variables respectively.

The IP model consists of $\frac{N^2 + N}{2}$ number of binary variables.
If we ignore the GSECs and the binary variable bounds, the total number of constraints characterizing the model add up to $N + 2$.

Note that the IP formulation only permits solutions visiting at least three vertices (the depot and two customers).
We could allow for single customer solutions by relaxing the binary variable requirements and by modifying the IP model.
Instead, it may be simpler to take a different approach and leave the IP model as is.
By employing a bruteforce algorithm in $\Theta(N)$ time we can scan for improving single-customer solutions.
After the resolution of the IP model, we check for improving single-customer solutions and, if any exist, we update the incumbent.

\subsection{GSECs inequalities}\label{sec:gsec-inequality}

\mytodo{Add more details/history/literature about GSECs here.}

It may be interesting to study the relationship between the number of non-zero binary variables participating in a GSEC as a function of the set size $\SetSize*{S}$.
To achieve this we can express the size of the edge crossing set $\delta(S)$ as

\begin{equation}\label{eq:delta-s-set-size}
	\SetSize*{\delta(S)} = \SetSize*{S} (N - \SetSize*{S})
\end{equation}

which is maximized when $\SetSize*{S} = \frac{N}{2}$, thus leading to the following easy upper bound:

\begin{equation}\label{eq:delta-s-set-size-ub}
	\SetSize*{\delta(S)} \le \frac{N^2}{4}
\end{equation}

Let $\NNZGSEC(S)$ denote the number of non-zero binary variables participating in a GSEC inequality.
It is easy to see that the following result holds
\begin{equation}\label{eq:gsec-nnz-ub}
	\NNZGSEC(S) \le 1 + \floor*{\frac{N^2}{4}} \quad \forall S \subseteq V_0,\ \SetSize*{S} \ge 2
\end{equation}
This trivial bound allows us to preallocate a fixed amount of memory beforehand prior to performing any inequality separation, saving us copious memory allocation time.

\subsection{Additional bounds}

\subsubsection{Demand lower bound}\label{sec:demand-lower-bound}
Although non strictly-necessary, it is possible to improve the linear relaxation of the IP model by introducing a static constraint modeling a lower bound on the minimum served demand:

\begin{equation}\label{eq:resource-lower-bound-constraint}
	\ExprCptpDemandSum[i]   \ge B
\end{equation}

where $B \in \R, B \ge 0$ is a constant and represents the the required minimum served demand.
The constant $B$ value can be computed as $B = q_u + q_v$, where $u, v \in V_0, u \ne v$ represent the least two demanding customers among all the customers, i.e. $q_u \le q_v \le q_i \quad \forall i \in V_0,\ i \ne u, v$.

\section{Additional Valid Inequalities}\label{sec:additional-valid-inequalities}

In this section we present additional valid inequalities for the CPTP problem.
These inequalities are not strictly required to solve a CPTP, but when embedded inside a Branch and Cut framework can heavily speed up the resolution process.
Most of these cuts follow the presentation proposed in \cite{jepsen2014}.
We will concentrate mostly on the most important ones, additional inequalities and information are treated in more detail in \cite{jepsen2014}.

\subsection{Generalized Large Multistar (GLM) inequalities}
The Generalized Large Multistar inequalities, GLM for short, were first proposed in \cite{gouveia1995}, and further discussed in \cite{letchford2006}.
In the additional work of \cite{letchford2002}, the GLM cuts are further generalized in the so-called Knapsack Large Multistar (KLM) inequalities.
The GLM inequalities can be expressed as:

\begin{equation}\label{eq:glm-inequality-v1}
	\begin{split}
		\ExprCptpFlowExiting{S} \ge \frac{2}{Q} \Expr*{  \ExprCptpDemandSumWithin[i]{S} + \ExprCptpDemandServedOutsideA{S} } \quad \forall S \subseteq V_0,\ \SetSize*{S} \ge 2
	\end{split}
\end{equation}

It is important to recall that we defined $q_0 = 0$.
Therefore, $\sum_{i \in S} \sum_{j \in V_0 \setminus S} q_j  x_{ij}$ becomes

\begin{equation}
	\ExprCptpDemandServedOutsideA{S} = \ExprCptpDemandServedOutsideB{S} =\ExprCptpDemandServedOutside{S}
\end{equation}

Finally, equation \eqref{eq:glm-inequality-v1} can be rewritten more concisely as:

\begin{equation}\label{eq:glm-inequality}
	\begin{split}
		\ExprCptpFlowExitingWithWeight{S}{\Expr*{1 - 2 \frac{q_j}{Q}}} - 2 	\ExprCptpServedDemandWithWeight{S}{i}{\frac{q_i}{Q}}  \ge  0   \quad \forall S \subseteq V_0,\ \SetSize*{S} \ge 2
	\end{split}
\end{equation}

Let $\NNZGLM(S)$ denote the number of non-zero binary variables participating in a GLM inequality as a function of the set $S$.
By using equation \eqref{eq:glm-inequality} and the result obtained in \eqref{eq:delta-s-set-size}, we have

\begin{equation}
	\NNZGLM(S) = -\SetSize*{S}^2 + (N + 1)\SetSize*{S}
\end{equation}

which is maximized when $\SetSize*{S} = \frac{N+1}{2}$.
It is now easy to see that the following result holds
\begin{equation}\label{eq:glm-nnz-ub}
	\NNZGLM(S) \le \floor*{ \frac{\Expr*{N + 1}^2}{4}} \quad \forall S \subseteq V_0,\ \SetSize*{S} \ge 2
\end{equation}

\subsection{Rounded Capacity Inequalities (RCI)}
The Rounded Capacity Inequalities, RCI for short, were first presented in \cite{achuthan1998}.

Let $q(S) = \sum_{i \in S} q_i$ and $\ExprQrS = \fmod{q(S)}{Q}$ be respectively the total demand and remainder capacity associated to set $S \subseteq V_0$.
The RCI inequalities can then be expressed as:

\begin{equation}\label{eq:rci-inequality}
	\begin{split}
		\ExprCptpFlowExiting{S} -2 \ExprCptpServedDemandWithWeight{S}{i}{\frac{q_i}{\ExprQrS}}   \ge   2 \left( \ceil*{ \frac{q(S)}{Q}} - \frac{q(S)}{\ExprQrS} \right) \quad \forall S \subseteq V_0
	\end{split}
\end{equation}

Let $\NNZGLM(S)$ denote the number of non-zero binary variables participating in a RCI inequality as a function of the set $S$.
The upper bound on $\NNZ_{\mt{RCI}}(S)$ follows the same reasoning of equation \eqref{eq:glm-nnz-ub}:

\begin{equation}\label{eq:rc-nnz-ub}
	\NNZRCI(S) \le \floor*{\frac{\left( N + 1\right)^2}{4}} \quad \forall S \subseteq V_0
\end{equation}

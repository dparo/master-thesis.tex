% memoir is a documentclass based on book.
% The memoir class incorporates the functions of a large number of commonly used packages
% to provide a more consistent interface, and is designed to be easy to customize.
% So instead of loading a dozen possibly conflicting packages---geometry to change margins,
% and setspace to change line spacing, and titlesec to format

% section headings, etc., etc.---you just specify memoir
% as your document class and use its settings



%%% IMPORTANT:    oneside   vs   twoside
%%%     twoside:
%%%        The twoside flag to memoir generates a PDF where even and odd pages are treated differently.
%%%        Specifically the margin is handled differently depending if it is an odd or even page.
%%%
%%%        PROS: The twoside flag is specifically useful for generating PDFs that should then be prepressed
%%%              into an actual physical book.
%%%              The first page of a chapter always starts in a odd page.
%%%        CONS: Digital reading of the pdf becomes awkard since when scrolling the text constantly
%%%              shifts left and right depending on the page index
%%%
%%%     oneside:
%%%         All pages are handled with the same margin, and are all treated the same
%%%
%%%         PROS: PDF better suited for digital reading on a computer
%%%         CONS: Prepressing on an actual book will lead to a lower quality result compared to the twoside.
%%%               Chapter first page can start at even an odd pages

%%% Memoir by default assumes twoside PDF if nothing is specified.
\documentclass[a4paper,twoside,11pt]{memoir}
\usepackage[dvipsnames]{xcolor}  % To use colors for specific texts, provides the \color{<COLOR>} to color text
\usepackage[utf8]{inputenc}
\usepackage[english]{babel}


\newbool{DebugBuild}
\setbool{DebugBuild}{false}

\chapterstyle{hansen}
\linespread{1.25}             % About 1.5 spacing in Word
\setlength{\parindent}{0pt}   % No paragraph indents
\setlength{\parskip}{1em}     % Paragraphs separated by one line


% How much section depth is actually gathered and shown in the table of contents
\setcounter{tocdepth}{2}       % 1 = show only section, 2 = show subsection, 3 = show subsubsection

% To which maximum section depth, latex produces a section numbering
\setcounter{secnumdepth}{5}       % For the Book document class
\setsecnumdepth{subsubsection}    % For the Memoir document class


\newcommand{\AcademicYear}{2020-2021}
\newcommand{\GraduationDay}{Jun, 2022}

\newcommand{\AuthName}{Davide}
\newcommand{\AuthSurname}{Paro}

\newcommand{\DegreeName}{Master Degree in Computer Engineering}
\newcommand{\Title}{A MIP based approach for the Capacitated Profitable Tour Problem (CPTP)}

\newcommand{\SupName}{Domenico}
\newcommand{\SupSurname}{Salvagnin}

%comment these lines if there are no supervisors
\newcommand{\CosupName}{Roberto}
\newcommand{\CosupSurname}{Roberti}

\newcommand{\UnivName}{University of Padova}
\newcommand{\UnivPlace}{Padova}
\newcommand{\DeptName}{Information Engineering}
\definecolor{M2}{HTML}{9B0014}
\definecolor{SchoolColor}{rgb}{0.71, 0, 0.106} %Unipd color





%%
%% Fix hyphenitazion of \texttt going over the margin
%%
% (added at TeX3)
% is used if TeX can not set the paragraph below the \tolerance badness,
% but rather than make overfull boxes it tries an extra pass "pretending" that every
% line has an additional \emergencystretch of stretchable glue, this allows the
% overall badness to be kept below 1000 and stops TeX "giving up" and putting
% all stretch into one line. So \emergencystretch does not change the setting of "good"
% paragraphs, it only changes the setting of paragraphs that would have produced
% over-full boxes. Note that you get warnings about the real badness calculation
% from TeX even though it retries with \emergencystretch the extra stretch
% is used to control the typesetting but it is not considered as good
% for the purposes of logging.
\emergencystretch 3em

%
%% "Core" packages
%%
\usepackage{ifthen}              % To write sane if-then statemnts
\usepackage{afterpackage}        % To run commands conditionally after a package is loaded (if it is loaded)
\usepackage[htt]{hyphenat}       % To enable hyphenation and provides hyphenatable monospaced fonts, i.e. \texttt{} reflows on new lines correctly
\usepackage{hyperref}            % To allow for interactive links (click with the mouse)
\usepackage{graphicx}            % Allows you to insert figures
\usepackage{amsmath}             % Allows you to do equations
\usepackage{amssymb}
\usepackage{mathtools}           % mathtools builds on top of asmmath
\usepackage{csquotes}
\usepackage{titling}
\usepackage{url}       % For typesetting urls
\usepackage{subcaption}    % For handling subfigures, i.e. nesting figures, list of figures, array of figures, etc

% biblatex is better than default bibtex
\usepackage[style=authoryear,autocite=plain]{biblatex}

%%
%% ""Debug"" packages
%%
\usepackage{todonotes}        % For writing todos and notes in yor pdf
\usepackage{lipsum}           % For Lorem Ipsum

\ifbool{DebugBuild} {
	\usepackage{showframe}        % To debug the frame border, padding, etc. It draws a box around each major region of the page
	\usepackage{layout}           % To generate automatically a layout page
} {
	% Else clause
}



%%
%% User level packages, can be customized to fit your need
%%
\usepackage{float}            % use [H] to force figure position exactly as placed in the original latex source
\usepackage{listings}         % For verbatim source code highlighting
\usepackage[linesnumbered, ruled]{algorithm2e}     % To write pseudocode algorithms in a consistent Latex-alike style

\usepackage{tikz}            % A drawing package to draw arbitrary shapes, usefull even for drawing networks/graphs
\usepackage{fancyvrb}        % Better verbatim environment, supporting also centering of the verbatim

\definecolor{webgreen}{rgb}{0,.5,0}
\definecolor{webbrown}{rgb}{.6,0,0}
\definecolor{Pantone}{RGB}{155,0,20}
\definecolor{GrigioLight}{RGB}{152, 152, 152}


\hypersetup{
	%hyperfootnotes=false,
	%pdfpagelabels,
	%draft,	% = elimina tutti i link (utile per stampe in bianco e nero)
	colorlinks=true,
	linktocpage=true,
	pdfstartpage=1,
	pdfstartview=FitV,
	% decommenta la riga seguente per avere link in nero (per esempio per la
	%stampa in bianco e nero)
	%colorlinks=false, linktocpage=false, pdfborder={0 0 0}, pdfstartpage=1,
	%pdfstartview=FitV,
	breaklinks=true,
	pdfpagemode=UseNone,
	pageanchor=true,
	pdfpagemode=UseOutlines,
	plainpages=false,
	bookmarksnumbered,
	bookmarksopen=true,
	bookmarksopenlevel=1,
	hypertexnames=true,
	pdfhighlight=/O,
	%nesting=true,
	%frenchlinks,
	urlcolor=webbrown,
	linkcolor=webbrown,
	citecolor=webgreen,
	%pagecolor=RoyalBlue,
	%urlcolor=Black, linkcolor=Black, citecolor=Black, %pagecolor=Black,
	pdftitle={\Title},
	pdfauthor={\textcopyright\ \AuthorName, \UnivName, \DeptName},
	pdfsubject={},
	pdfkeywords={},
	pdfcreator={pdfLaTeX},
	pdfproducer={LaTeX}
}

\captionsetup{
	tableposition=top,
	figureposition=bottom,
	font=small,
	format=hang,
	labelfont=bf
}

% Adds comma in in-text citations
\renewcommand*{\nameyeardelim}{\addcomma\space}


\newcommand{\myinputchap}[1]{\input{#1}\clearpage}

\DeclarePairedDelimiter\ceil{\lceil}{\rceil}
\DeclarePairedDelimiter\floor{\lfloor}{\rfloor}

\newcommand{\N}{\mathbb N}
\newcommand{\Z}{\mathbb Z}
\newcommand{\Q}{\mathbb Q}
\newcommand{\R}{\mathbb R}

% Shorten \mathrr for equation: this is useful to define multi-character variables in mathemtical equations
\newcommand{\mt}[1]{\mathrm{#1}}

\newcommand{\mytodo}[1]{\todo[inline]{TODO: #1}}


\newcommand{\lorem}{\lipsum[1]}

% algorithm2e package customization
%   Define \Comment command inside \begin{algorithm}\end{...} environment to define comments
\SetKwComment{Comment}{/* }{ */}
\SetKw{Assert}{assert}
\SetKwRepeat{Do}{do}{while}          % Define do/while construct


\addbibresource{biblio.bib}
\addbibresource{ro2.bib}


\begin{document}

% turns off chapter numbering and uses roman numerals for page numbers
\frontmatter

%%
%% Handle titlepage: Eg the first page which is usually fancy with lots of university logos
%%
\newcommand\ThemeTitlePage{themes/\ThemeName/titlepage.tex}

\IfFileExists{\ThemeTitlePage}{
	\input{\ThemeTitlePage}
	\cleardoublepage
}{}


\IfFileExists{Content/Preamble/colophon.tex}{
	\hfill
\vfill

{
	\setlength\parindent{0pt}

    \AuthorName\ \AuthorSurname: \textit{\Title}, \DegreeName, \textcopyright\ \CopyrightYear.
    \par\hfill{\scriptsize\color{gray}Last updated on \today}
}

	\clearpage
}{}


\IfFileExists{Content/Preamble/inscription.tex}{
	\phantomsection
\thispagestyle{empty}
\pdfbookmark{Inscription}{Inscription}

\vspace*{3cm}

\begin{flushright}{
		\slshape Do not shorten the morning by getting up late, or waste it in
		unworthy occupations or in talk; look upon it as the quintessence of life, as
		to a certain extent sacred. Evening is like old age: we are languid,
		talkative, silly. Each day is a little life: every waking and rising a little
		birth, every fresh morning a little youth, every going to rest and sleep a
		little death.} \\ \medskip Arthur Schopenhauer
\end{flushright}


\medskip

	\clearpage
}{}


\IfFileExists{Content/Preamble/acknowledgements.tex}{
	\vspace{1.0cm}
{
	\setlength\parindent{0pt}

	Vorrei innanzitutto ringraziare la mia famiglia per il sostegno datomi e per aver sempre creduto in me.
	È solo grazie a loro che ora mi trovo qui, alla fine di questo tortuoso percorso di studi.
	Sono entusiasta di aver raggiunto questi obbiettivi, nemmeno il me stesso di qualche anno fa ci avrebbe creduto,
	e invece eccomi qui, ed è solo grazie a voi.
	Grazie mamma, papà e Massimo.
	Vi voglio bene.
	Ringrazio la mia compagna Nicla, per l'affetto e per gli ultimi bellissimi due anni universitari passati assieme.
	Mi hai aiutato in momenti difficili, fornendomi consigli e aiutandomi a prendere decisioni.
	Mi riempe di gioia ripensare ai nostri incontri tra le strade padovane durante le pause tra una lezione e l'altra.
	Oppure le nostre lunghe (circa) sessioni di studio nelle aule Taliercio e Marsala.
	Oppure ancora le nostre serate "pizzata" del Venerdì.
	Fantastici ricordi.
	Colgo l'occasione per ringraziare anche Emanuela, Claudio e nonna Flavia per l'ospitalità
	e cordialità fornitami.
	Ringrazio la struttura Murialdo per avermi ospitato durante la mia carriera universitaria e
	i Murialdini che sono stati ottimi compagni di cena, di uscite serali e di calcetto.
	Già che ci sono ringrazio pure la mensa Murialdo, rimani sempre la migliore.

	\medskip

	Infine, vorrei ringraziare i miei supervisori Domenico Salvagnin e Roberto Roberti,
	a cui porgo la mia gratitude, sia per l'aiuto fornito durante la stesura di questo lavoro,
	sia per avermi permesso di studiare un problema di ricerca interessante.

	I would also like to thank Ruslan Sadykov for his generous assistance in setting up \bapcod{} and reviewing the C++ code of the implemented model.

}

%%%%%%%%%%%%%%%%%%%%%%%%%%%%%%%%%%%%%%%%%%%%
%%%%%%%%%%%%%%%%%%%%%%%%%%%%%%%%%%%%%%%%%%%%
%%%%%%%%%%%%%%%%%%%%%%%%%%%%%%%%%%%%%%%%%%%%

\vspace{1cm}

{
	\setlength\parindent{0pt}

	\textit{\UnivPlace}\\
	\textit{\GraduationDate}    \hfill    \AuthorName{} \AuthorSurname{}
}

	\clearpage
}{}



% %%
% %% Include the abstract if any
% %%
\IfFileExists{Content/Preamble/abstract.tex}{
	\pdfbookmark{Abstract}{Abstract}
	\begin{abstract}
		\noindent The \textit{Capacitated Vehicle Routing Problem}, CVRP for short,
is a combinatorial optimization routing problem in which,
a geographically dispersed set of customers with known demands
must be served by a fleet of vehicles stationed at a central facility.
\textit{Column generation} techniques embedded within \textit{branch-price-and-cut} frameworks
have been the de facto state-of-the-art dominant approach
for building exact algorithms for the CVRP over the last two decades.
The \textit{pricer}, a critical component in column generation, must solve
the \textit{Pricing Problem} (PP), which asks for an
\textit{Elementary Shortest Path Problem with Resource Constraints} (ESPPRC)
in a reduced-cost network.
Little scientific efforts have been dedicated to studying
branch-and-cut based approaches for tackling the PP.

The ESPPRC has been traditionally relaxed and solved through dynamic programming algorithms.
This approach, however, has two major drawbacks.
For starters, it worsens the obtained dual bounds.
Furthermore, the running time degrades as the length of the generated routes increases.
To evaluate the performance of their contributions, the operations research community has traditionally used a set of historical and artificial test instances.
However, these benchmark instances do not capture the key characteristics of modern real-world distribution problems, which are usually characterized by longer routes.

\noindent In this thesis, we develop
a scheme based on a branch-and-cut approach for solving the pricing problem.
We study the behavior and effectiveness of our implementation in producing longer routes by comparing it with state-of-the-art solutions based on dynamic programmingalgorithms.
Our results suggest that branch-and-cut approaches may supplement the traditional labeling algorithm, indicating that further future research in this area may bring benefits to CVRP solvers.

	\end{abstract}
	\cleardoublepage
}{}

\IfFileExists{Content/Preamble/abstract_ita.tex}{
	\pdfbookmark{Sommario}{Sommario}
	\begin{otherlanguage}{italian}
		\begin{abstract}
			{
\setlength\parindent{0pt}

Il \textit{Capacitated Vehicle Routing Problem}, abbreviato come CVRP,
è un problema di ottimizzazione combinatoria d'instradamento nel quale,
un insieme geograficamente sparso di clienti con richieste note
deve essere servito da una flotta di veicoli stazionati in una struttura centrale.
Negli ultimi due decenni,
tecniche di \textit{Column generation} incorporate all'interno di frameworks \textit{branch-price-and-cut}
sono state infatti l'approccio stato dell'arte dominante per la costruzione di algoritmi esatti per il CVRP.
Il \textit{pricer}, un componente critico nella column generation, deve risolvere
il \textit{Pricing Problem} (PP) che richiede la risoluzione di un
\textit{Elementary Shortest Path Problem with Resource Constraints} (ESPPRC)
in una rete di costo ridotto.
Pochi sforzi scientifici sono stati dedicati allo studio di approacci
branch-and-cut per affrontare il PP.

L'ESPPRC è stato tradizionalmente rilassato e risolto attraverso algoritmi di programmazione dinamica.
Questo approccio, tuttavia, ha due principali svantaggi.
Per cominciare, peggiora i dual bounds ottenuti.
Inoltre, il tempo di esecuzione diminuisce all'aumentare della lunghezza dei percorsi generati.
Per valutare la performance dei loro contributi,
la comunità di ricerca operativa ha tradizionalmente utilizzato una serie di istanze
di test storiche e artificiali.
Tuttavia, queste istanze di benchmark non catturano le caratteristiche chiave dei moderni problemi di distribuzione del mondo reale, che sono tipicamente caraterrizati da lunghi percorsi.

In questa tesi sviluppiamo
uno schema basato su un approaccio branch-and-cut per risolvere il pricing problem.
Studiamo il comportamento e l'efficacia della nostra implementazione nel produrre percorsi più lunghi comparandola con soluzioni all'avanguardia basate su programmazione dinamica.
I nostri risultati suggeriscono che gli approcci branch-and-cut possono supplementare il tradizionale algoritmo di etichettatura, indicando che ulteriore ricerca in quest'area possa portare benefici ai risolutori CVRP.
}

		\end{abstract}
	\end{otherlanguage}

	\cleardoublepage
}{}


%%
%% Produce table of contents
%%
\ifbool{GenerateTableOfContents}{
	\tableofcontents
	\cleardoublepage
}{}

%%
%% Produce list of figures
%%
\ifbool{GenerateListOfFigures}{
	\listoffigures
	\cleardoublepage
}{}

%%
%% Produce list of tables
%%
\ifbool{GenerateListOfTables}{
	\listoftables
	\cleardoublepage
}{}

\pagebreak

% Turns on arabic numbering for the main content
\pagenumbering{arabic}

\mainmatter
%%%%%%%%%%%%%%%%%%%%%%%%%%%%%%%%%%%%%%%%%%%%%%%%%%%%%%%%%%%%%%%%%%%%%%%%%%%%%%%%%%%%%%%%%%%%%%%%
%%%%%%%%%%%%%%%%%%%%%%%%%%%%%%%%%%%%%%%%%%%%%%%%%%%%%%%%%%%%%%%%%%%%%%%%%%%%%%%%%%%%%%%%%%%%%%%%
\mainmatter % turns on chapter numbering, resets page numbering and uses arabic numerals for page numbers

\myinputchap{Content/Chapters/chapter1.tex}
\myinputchap{Content/Chapters/example-chapter.tex}

%%%%%%%%%%%%%%%%%%%%%%%%%%%%%%%%%%%%%%%%%%%%%%%%%%%%%%%%%%%%%%%%%%%%%%%%%%%%%%%%%%%%%%%%%%%%%%%%
%%%%%%%%%%%%%%%%%%%%%%%%%%%%%%%%%%%%%%%%%%%%%%%%%%%%%%%%%%%%%%%%%%%%%%%%%%%%%%%%%%%%%%%%%%%%%%%%
\appendix % resets chapter numbering, uses letters for chapter numbers and doesn't fiddle with page numbering

\myinputchap{Content/Appendix/appendix_a.tex}

\pagebreak


\backmatter
\printbibliography

\end{document}
